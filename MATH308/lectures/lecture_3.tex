% !TeX root = ../diffeq.tex
\documentclass[diffeq.tex]{subfiles}

\begin{document}
\chapter{21 January 2020}
    \section{Linear Differential Equations (cont).}
    \begin{example}
        Given $y'-2y=t^{2}e^{2t}$ find:
        \begin{enumerate}
            \item The general solution
            \begin{align}
                p(t) = -2,&\ g(t) = t^{2}e^{2t}\\
                \mu(t) %&= \exp\bbbp{\int p(t)\,dt}\\
                &= \exp\left(\int -2\,dt\right)\\
                &= e^{-2t + c}\\
                y_{c}(t) &= e^{2t}\int t^{2}\,dt\\
                &= e^{2t}\left(\frac{1}{3}t^{3} + c\right)
            \end{align}
            \item What is $\lim_{t \rightarrow \infty} y_{c}(t)$?\\
            There are infinitely many $y_{c}(t)$; the answer may vary with the value of $c$. In this case, the value of $c$ does not matter.
            $$\lim_{t \rightarrow \infty} y_{c}(t) = +\infty$$
        \end{enumerate}
    \end{example}
    \np
    \section{Separable Differential Equations}
    \begin{definition}[Separable Differential Equations]
        A \textbf{separable differential equation} (SDE) can be defined by
        \begin{equation}
            \frac{dy}{dx}=y'=f(x, y)=-\frac{M(x, y)}{N(x, y)}
        \end{equation}
        where
        \begin{align}
            M(x, y) &= - f(x, y)\\
            N(x, y) &= 1
        \end{align}
        it is \textbf{separable} because it can be written in the \textbf{differential form}
        \begin{equation}
            M(x)\,dx+N(y)\,dy=0
        \end{equation}
    \end{definition}
    \begin{btheorem}
        If $\frac{dy}{dx} = \frac{M(x)}{N(y)}$, then $\int N(y)\,dy=\int M(x)\,dx$
    \end{btheorem}
    \begin{bproof}
        Choose $\widetilde{N}$ such that $\frac{d\widetilde{N}(y)}{dx} = M(x)$:
    \begin{align}
        \frac{d\widetilde{N}(y)}{dy} = \frac{d\widetilde{N}(y)}{dx}\frac{dx}{dy} &= \frac{d\widetilde{N}(y)}{dy}\frac{dy}{dx}= \frac{d\widetilde{N}(x)}{dx}\\
        \frac{d\widetilde{N}(y)}{dy} &= \frac{dy}{dx}
        \\
        \implies \frac{d\widetilde{N}(y)}{dx} &= M(x)
    \end{align}
    \end{bproof}
    \np
    \begin{example}
        Find a particular solution that passes through the point $(0, 1)$.
        \begin{align}
            \frac{dy}{dx} &= \frac{4x-x^{3}}{4 + y}\\
            \implies \int (4 + y)\,dy &= \int(4x-x^{3})\,dx\\
            4y+\frac{1}{2}y^{2} + c_1 &= 2 x^{2}-\frac{1}{4}x^{4} + c_2\\
            4y + \frac{1}{2} y^{2} &= 2 x^{2} - \frac{1}{4} x^{4} + (c_2 - c_1)\\
            \implies 2y + 16y + x^{4} &- 8x^{2} + c = 0\\
            (0, 1) \implies 2(1) + 16(1) + 0^{4} &- 8(0)^{2} + c = 0\\
            c &= -18\\
            \therefore 2y + 16y &+ x^{4} - 8x^{2} = 18
        \end{align}
    \end{example}
    \begin{homework}
        \begin{align}
            y' = \frac{dy}{dx} &= \frac{x^{2}}{y}\\
            y\,dy &= x^{2}\,dx\\
            \int y\,dy &= \int x^{2}\,dx\\
            \frac{1}{2}y^{2} &= \frac{1}{3}x^{3} + c\\
            y(x) &= \pm \sqrt{\frac{2}{3}x^{3} + c}
        \end{align}
    \end{homework}
\end{document}