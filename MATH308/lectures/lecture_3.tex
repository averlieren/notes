% !TeX root = ../diffeq.tex
\documentclass[diffeq.tex]{subfiles}

\begin{document}
\chapter{21 January 2020}
    \section{Linear Differential Equations (cont.)}
    \begin{example}
        Find the general solution and the limit of the solution as $t \to \infty$
        \begin{equation}
            y' - 2y = t^{2}e^{2t}
        \end{equation}
        \textbf{Solution:}\\
        The general solution is as follows
        \begin{alignat}{1}
            \mu(t) &= e^{\int -2\,dt}\\
            &= e^{-2t}\\
            y_{c} &= e^{2t}\int t^{2}\,dt\\
            &= e^{2t}\left(\frac{1}{3}t^{3} + c\right)
        \end{alignat}
        Then, we can take the limit of $y_{c}$
        \begin{equation}
            \lim_{t \to \infty} e^{2t}\left(\frac{1}{3}t^{3} + c\right) = \infty
        \end{equation}
    \end{example}
    \np
    \section{Separable Differential Equations}
    Notice that a general first order ODE is given by
    \begin{equation}
        y' = f(t, y)
    \end{equation}
    What if we let $M(t, y) = -f(t, y)$ and $N(t, y) = 1$?
    \begin{equation}
        M(t, y) + N(t, y)y' = 0
    \end{equation}
    Then,
    \begin{equation}
        N(t, y)\,dy = -M(t, y)\,dt
    \end{equation}
    And finally,
    \begin{equation}
        M(t)\,dt + N(y)\,dy = 0
    \end{equation}
    The above form is considered to be the \textbf{differential form}.
    \begin{definition}[Separable DE]
        A \textbf{separable DE} can be written in the following form
        \begin{equation}
            y' = -\frac{M(t, y)}{N(t, y)} = \frac{M(t, y)}{N(t, y)}
        \end{equation}
        It is called separable because when written in the \textbf{differential form}, the terms of each variable can be separated to opposite sides of the equation.\\
        Note that we can write it without the negative sign, this is because we can redefine either functions $M$ or $N$ to include the negative.
    \end{definition}
    \np
    \begin{theorem}
        If
        \begin{equation}
            \frac{dy}{dt} = \frac{M(t)}{N(y)}
        \end{equation}
        then
        \begin{equation}
            \int N(y)\,dy=\int M(t)\,dt
        \end{equation}
    \end{theorem}
    \begin{proof}
        (This course is not proof based, so proofs will not be rigorous)\\
        Choose $\widetilde{N}$ such that $\frac{d}{dt}\widetilde{N}(y) = M(t)$.
        \begin{equation}
            \frac{df(y)}{dy} = \frac{df(y)}{dt}\frac{dt}{dy} = \frac{df(t)}{dy}\frac{dy}{dt} = \frac{df(t)}{dt}
        \end{equation}
        From the above, we can conclude the following results
        \begin{equation}
            \frac{d}{dy}\widetilde{N}(y) = \frac{dy}{dt},\ \frac{d}{dt}\widetilde{N}(y) = M(t)
        \end{equation}
    \end{proof}
    \begin{example}
        Find a particular solution that passes through the point $(0, 1)$.
        \begin{equation}
            \frac{dy}{dx} = \frac{4x - x^{3}}{4 + y}
        \end{equation}
        \textbf{Solution:}
        \begin{alignat}{1}
            \int(4 + y)\,dy &= \int(4x - x^{3})\,dx\\
            4y + \frac{1}{2}y^{2} + c_{1} &= 2x^{2} - \frac{1}{4}x^{4} + c_{2}
        \end{alignat}
        Notice that we can combine the constants $c_{1}$ and $c_{2}$.
        \begin{equation}
            4y + \frac{1}{2}y^{2} = 2x^{2} - \frac{1}{4}x^{4} + c
        \end{equation}
        We can rearrange to bring the variables over to one side and then solve for $c$ using the point $(0, 1)$.
        \begin{alignat}{1}
            2y + 16y + x^{4} -8x^{2} + c &= 0\\
            2 + 16 + c &= 0\\
            c &= -18
        \end{alignat}
        Then, the final solution
        \begin{equation}
            2y + 16y + x^{4} - 8x^{2} - 18 = 0
        \end{equation}
    \end{example}
    \begin{homework}
        Solve
        \begin{equation}
            y' = \frac{x^{2}}{y}
        \end{equation}
        \textbf{Solution:}
        \begin{alignat}{1}
            \frac{dy}{dx} &= \frac{x^{2}}{y}\\
            y\,dy &= x^{2}\,dx\\
            \int y\,dy &= \int x^{2}\,dx\\
            \frac{1}{2}y^{2} &= \frac{1}{3}x^{3} + c
        \end{alignat}
        Notice that we can solve explicitly rather than implicitly, giving us
        \begin{equation}
            y(x) = \pm \sqrt{\frac{2}{3}x^{3} = c}
        \end{equation}
    \end{homework}
\end{document}