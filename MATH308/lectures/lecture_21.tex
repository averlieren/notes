% !TeX root = ../diffeq.tex
\documentclass[diffeq.tex]{subfiles}

\begin{document}
\chapter{9 April 2020}
    \section{Linear Algebra}
    \subsection{Inverse Matrices}
    \begin{definition}[Identity Matrix]
        An identity matrix of size $n$ is defined as an $n$\,x\,$n$ matrix whose main diagonal is filled with $1$s and all other entries are $0$,
        \begin{equation}
            I_{n} = \begin{pmatrix}
                1 & 0 & \dots & 0\\
                0 & 1 & \dots & 0\\
                \vdots & \vdots & \ddots & \vdots \\
                0 & 0 & \dots & 1 
            \end{pmatrix}
        \end{equation}
        It is called either the identity matrix or the unit matrix, however, the latter is much less common. It is called as such because it satisfies the following identity
        \begin{equation}
            I\,A = A\,I = A
        \end{equation}
    \end{definition}
    \begin{definition}[Inverse Matrix]
        For a square matrix $A$ of size $n$\,x\,$n$, then the matrix $B$ also of size $n$\,x\,$n$ is the inverse matrix if it satisfies the following identity:
        \begin{equation}
            A\,B = I_{n}
        \end{equation}
        The inverse matrix is not defined for all matrices, take the zero matrix for example.
    \end{definition}
    \begin{theorem}
        A square matrix, $A$, has an inverse iff $\det(A) \neq 0$.
    \end{theorem}
    \begin{example}
        \begin{equation}
            A = \begin{pmatrix}
                0 & 1\\
                0 & 2
            \end{pmatrix},\quad \det(A) = 0 \implies \nexists A^{-1}
        \end{equation}
        \begin{equation}
            A = \begin{pmatrix}
                0 & 1\\
                -1 & 2
            \end{pmatrix},\quad \det(A) = 1 \implies \exists A^{-1}
        \end{equation}
    \end{example}
    \begin{example}[Finding the Inverse Matrix]
        The inverse matrix of $A$ can be found via the augmented matrix of $A$ and $I$,
        \begin{equation}
            A = \begin{pmatrix}
                1 & 2 & 3\\
                0 & -1 & 0\\
                1 & 3 & 2
            \end{pmatrix}
        \end{equation}
        \begin{equation}
            \left(\begin{array}{@{}ccc|ccc@{}}
                1 & 2 & 3 & 1 & 0 & 0 \\
                0 & -1 & 0 & 0 & 1 & 0 \\
                1 & 3 & 2 & 0 & 0 & 1
                \end{array}\right)
        \end{equation}
        Then, to compute the inverse is to make the left hand size of the augmented matrix look like the identity matrix on the right hand size.
        \begin{enumerate}
            \item $R_{3} - R_{1}$
            \begin{equation}
                \left(\begin{array}{@{}ccc|ccc@{}}
                    1 & 2 & 3 & 1 & 0 & 0 \\
                    0 & -1 & 0 & 0 & 1 & 0 \\
                    0 & 1 & -1 & -1 & 0 & 1
                    \end{array}\right)
            \end{equation}
            \item $R_{1} - 2R_{3}$
            \begin{equation}
                \left(\begin{array}{@{}ccc|ccc@{}}
                    1 & 0 & 5 & 3 & 0 & -2 \\
                    0 & -1 & 0 & 0 & 1 & 0 \\
                    0 & 1 & -1 & -1 & 0 & 1
                    \end{array}\right)
            \end{equation}
            \item $R_{3} + R_{2}$
            \begin{equation}
                \left(\begin{array}{@{}ccc|ccc@{}}
                    1 & 0 & 5 & 3 & 0 & -2 \\
                    0 & -1 & 0 & 0 & 1 & 0 \\
                    0 & 0 & -1 & -1 & 1 & 1
                    \end{array}\right)
            \end{equation}
            \item $R_{1} + 5R_{3}$
            \begin{equation}
                \left(\begin{array}{@{}ccc|ccc@{}}
                    1 & 0 & 0 & -2 & 5 & 3 \\
                    0 & -1 & 0 & 0 & 1 & 0 \\
                    0 & 0 & -1 & -1 & 1 & 1
                    \end{array}\right)
            \end{equation}
            \item $R_{2} - 2R_{2}$ and $R_{3} - 2R_{3}$
            \begin{equation}
                \left(\begin{array}{@{}ccc|ccc@{}}
                    1 & 0 & 0 & -2 & 5 & 3 \\
                    0 & 1 & 0 & 0 & -1 & 0 \\
                    0 & 0 & 1 & 1 & -1 & -1
                    \end{array}\right)
            \end{equation}
        \end{enumerate}
        Then confirm that the final result is the inverse matrix,
        \begin{equation}
            \begin{pmatrix}
                1 & 2 & 3\\
                0 & -1 & 0\\
                1 & 3 & 2
            \end{pmatrix}
            \begin{pmatrix}
                -2 & 5 & 3 \\
                0 & -1 & 0 \\
                1 & -1 & -1
            \end{pmatrix} = \begin{pmatrix}
                1 & 0 & 0 \\
                0 & 1 & 0 \\
                0 & 0 & 1
            \end{pmatrix}
        \end{equation}
    \end{example}
    \subsection{Eigenvectors and Eigenvalues}
    First some review over matrices, a matrix ($m$\,x\,$n$) multiplied by a vector ($n$\,x\,$1$), i.e.
    \begin{equation}
        A = \begin{pmatrix}
            a_{11} & a_{12} & \dots & a_{1n}\\
            a_{21} & a_{22} & \dots & a_{2n}\\
            \vdots & \vdots & \ddots & \vdots\\
            a_{m1} & a_{m2} & \dots & a_{mn}
        \end{pmatrix},\quad \bar{x} = \begin{pmatrix}
            x_{1}\\
            x_{2}\\
            \vdots\\
            x_{n}
        \end{pmatrix}
    \end{equation}
    Then,
    \begin{equation}
        A\,\bar{x} = \begin{pmatrix}
            x_{1}a_{11} + x_{2}a_{12} + \dots + x_{n}a_{1n}\\
            x_{1}a_{21} + x_{2}a_{22} + \dots + x_{n}a_{2n}\\
            \vdots \\
            x_{1}a_{m1} + x_{2}a_{m2} + \dots + x_{n}a_{mn}
        \end{pmatrix}
    \end{equation}
    A system of equations can be written as follows
    \begin{equation}
        \left\{\begin{array}{@{}cccc@{}}
            2x &+3y &+ z &= 1\\
            &4y &-z &= 0\\
            4x &&-5z &= 12
            \end{array}\right.
        \iff
        \begin{pmatrix}
            2 & 3 & 1\\
            0 & 4 & -1\\
            4 & 0 & 5
        \end{pmatrix}
        \begin{pmatrix}
            x \\ y \\ z
        \end{pmatrix}
        =
        \begin{pmatrix}
            1 \\ 0 \\ 12
        \end{pmatrix}
    \end{equation}
    \begin{definition}[Eigenvectors and Eigenvalues]
        Let $A$ be a square ($n$\,x\,$n$) matrix, then $\bar{x}$ as a vector with $n$ elements is an \textbf{eigenvector} if $\exists \lambda, \lambda \neq 0$, such that
        \begin{equation}
            (A - \lambda I)\bar{x} = 0
        \end{equation}
        The value of $\lambda$ is defined to be the \textbf{eigenvalue}.
        Generally, there are $n$ eigenvalues, and infinitely many eigenvectors for each eigenvalue.
    \end{definition}
    \begin{remark}[Note about Eigenvectors]
        Although the zero matrix satisfies
        \begin{equation}
            (A - \lambda I)\bar{x} = 0
        \end{equation}
        it is not considered to be an eigenvector, therefore
        \begin{equation}
            \bar{x} \neq 0
        \end{equation}
    \end{remark}
    \begin{theorem}[Eigenvalue]
        $\lambda$ is an eigenvalue of a matrix $A$, if and only if
        \begin{equation}
            \det(A - \lambda I) = 0
        \end{equation}
    \end{theorem}
    \begin{example}
        Find the eigenvalues and eigenvectors of
        \begin{equation}
            A = \begin{pmatrix}
                3 & -1\\
                4 & -2
            \end{pmatrix}
        \end{equation}
        Solution: Find $\lambda$ such that, $|A - \lambda I| = 0$.
        \begin{alignat}{1}
            |A - \lambda I| &= \begin{vmatrix}
                3 - \lambda & 1\\
                4 & -2-\lambda
            \end{vmatrix} = 0\\
            &= (3 - \lambda)(-2-\lambda) - (-1)(4)\\
            0 &= \lambda^{2} - \lambda - 2
        \end{alignat}
        Solving the quadratic yields $\lambda_{1} = -1, \lambda_{2} = 2$.
        \begin{enumerate}
            \item Finding the eigenvectors for $\lambda_{1}$.
            \begin{equation}
                (A - \lambda I)\bar{x} = 0 \iff \begin{cases}
                    4x_{1} - x_{2} = 0\\
                    4x_{1} - x_{2} = 0
                \end{cases} \implies \bar{x} = \begin{pmatrix}
                    1\\
                    4
                \end{pmatrix}
            \end{equation}
            Therefore, the eigenvectors for $\lambda = -1$ are
            \begin{equation}
                c\bar{x} = c\begin{pmatrix}
                    1\\
                    4
                \end{pmatrix}, c \neq 0
            \end{equation}
        \end{enumerate} % TODO add eigenvector when lambda = 2
    \end{example}
\end{document}