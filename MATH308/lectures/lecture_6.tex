% !TeX root = ../diffeq.tex
\documentclass[diffeq.tex]{subfiles}

\begin{document}
\chapter{30 January 2020}
    \section{Exact Differential Equations (cont.)}
    \begin{example}
        Solve
        \begin{equation}
            \label{Exact1}
            (y\cos x + 2xe^{y}) + (\sin x + x^{2} + x^{2}e^{y} - 1)y' = 0
        \end{equation}
        Checking if \textbf{Equation \ref{Exact1}} is exact,
        \begin{alignat}{2}
            \frac{\partial M(x, y)}{\partial y} &= \frac{\partial (y\cos x + 2xe^y)}{\partial y} &= \cos x + 2xe^y\\
            \frac{\partial N(x, y)}{\partial x} &= \frac{\partial (\sin x + x^{2}e^{y} - 1)}{\partial y} &= \cos x + 2xe^y
        \end{alignat}
        From the above, this is an exact differential equation.
        \begin{alignat}{1}
            \psi(x, y) &= \int M(x, y)\,dx + h(y)\\
            &= \int (y\cos x + 2xe^{y})\,dx + h(y)\\
            &= h(y) + y\sin x+ x^{2}e^{y} + C\\
            &= h(y) + y\sin x + x^{2}e^{y}
         \end{alignat}
         Notice that the constant can be neglected as it can be contained in $h(y)$. Now solving for $h(y)$,
         \begin{alignat}{3}
            &&\psi_{y}(x, y) &= N(x, y)\\
            &\implies&\frac{dh}{dy} + \frac{\partial(y\sin x + e^{y}x^{2})}{\partial y}&=\sin x + x^{2}e^{y} - 1\\
            &&\frac{dh}{dy} + \sin x + x^{2}e^{y}&=\sin x + x^{2}e^{y} - 1\\
            &&\frac{dh}{dy}&= -1\\
            &&h&=-y + C
         \end{alignat}
         Then,
         \begin{equation}
             \psi(x, y) = y\sin x + x^{2}e^{y} - y + C
         \end{equation}
         Finally, $y(x)$ is given by the implicit expression
         \begin{equation}
             y\sin x + x^{2}e^{y} - y = C
         \end{equation}
    \end{example}
    \begin{example}
        Solve
        \begin{equation}
            \label{Exact2}
            (3xy + y^{2}) + (x^{2} + xy)y' = 0
        \end{equation}
        Checking if the equation is exact,
        \begin{alignat}{1}
            \frac{\partial (3xy + y^{2})}{\partial y} &= 3x + 2y\\
            \frac{\partial (x^{2} + xy)}{\partial x} &= 2x + y
        \end{alignat}
        Notice that they are not equal; however,
        \begin{equation}
            \mu(x)(3xy + y^{2}) + \mu(x)(x^{2} + xy)y' = 0
        \end{equation}
        is an exact differential equation if
        \begin{equation}
            -\frac{N_{x}(x, y) + M_{y})x, y}{N(x, y)}
        \end{equation}
        is a function dependent only on $x$. However, $\forall M(x, y), N(x, y) \nexists \mu(x)$. $\mu(x)$ can be found by solving the differential equation,
        \begin{alignat}{1}
            \frac{d\mu}{dx} &= \frac{-N_{x}(x, y) + M_{y}(x, y)}{N(x, y)}\mu\\
            \mu(x) &= \exp\bbbp{\int\frac{N_{x} - M_{y}}{N}\,dx}
        \end{alignat}
        In this problem,
        \begin{equation}
            \frac{M_{y} - N_{x}}{N} = \frac{(3x + 2y) - (2x + y)}{x^2 + xy} = \frac{1}{x}
        \end{equation}
        \begin{equation}
            \mu(x) = \exp\bbbp{\int\frac{dx}{x}} = x + C
        \end{equation}
        Multiplying \textbf{Equation \ref{Exact2}} by $\mu(x)$ yields,
        \begin{equation}
            (3x^{2}y + y^{2}x) + (x^{3} + x^{2}y)y' = 0
        \end{equation}
        Checking if the equation is exact yields the following,
        \begin{alignat}{1}
            \frac{\partial 3x^{2}y + xy^{2}}{\partial y} &= 3x^{2} + 2xy\\
            \frac{\partial x^{3} + x^{2}y}{\partial x} &= 3x^{2} + 2xy
        \end{alignat}
        and is therefore exact.
        \begin{alignat}{1}
            \psi(x, y) &= \int(3x^{2}y + xy^{2})\,dx + h(y)\\
            &= x^{3}y + \frac{1}{2}x^{2}y^{2}+h(y)\\
            \frac{\partial\psi(x, y)}{\partial y} &= x^{3} + x^{2}y + \frac{dh}{dy}\\
            &= N(x, y)\\
            \frac{dh}{dy} &= x^{3} + x^{2}y = x^{3} + x^{2}y\label{6232}\\
            h &= 0
        \end{alignat}
        Finally, $y(x)$ can be expressed as,
        \begin{equation}
            x^{3}y + \frac{1}{2}x^{2}y^{2} = C
        \end{equation}
    \end{example}
    \begin{method}[Solving Exact Differential Equations]
        \begin{enumerate}
            \item Step 1: Determine if the equation is exact
            \begin{equation}
                \frac{\partial M(x, y)}{\partial y} = \frac{\partial N(x, y)}{\partial x}
            \end{equation}
            \item Step 2: Find $\psi(x, y)$ such that $\psi_{x}(x, y) = M(x, y)$, and $\psi_{y}(x, y) = N(x, y)$. Generally,
            \begin{equation}
                    \psi(x, y) = \int M(x, y)\,dx + h(y)
            \end{equation}
            this works because
            \begin{alignat}{1}
                \frac{\partial \psi(x, y)}{\partial x} &= \frac{\partial \int M(x, y)\,dx}{\partial x} + \frac{\partial h(y)}{\partial x}\\
                &= M(x, y) + 0 
            \end{alignat}
            Then find $h(y)$ such that $\psi_{y}(x, y) = N(x, y)$.
        \end{enumerate}
    \end{method}
    \begin{remark}
        Note in step 2 of \textbf{Method 6.1}
        \begin{equation}
            \psi(x, y) = \int M(x, y)\,dx + h(y)
        \end{equation}
        can also be defined as
        \begin{alignat}{1}
            \psi(x, y) &= \int N(x, y)\,dy + h(x)\\
            \frac{\partial \psi(x, y)}{\partial y} = \psi_{y}(x, y) &= \frac{\partial \int N(x, y)\,dy}{\partial y} + \frac{\partial h(x)}{\partial y}\\
            &= N(x, y) + 0
        \end{alignat}
    \end{remark}
    \begin{remark}
        $y(x)$ is a solution for $M(x, y)\,dx + N(x, y)\,dy = 0$ iff $\psi(x, y(x)) = c$.
        Consider the following,
        \[
            \frac{d\psi(f_{1}, f_{2})}{dt} = \frac{\partial \psi(x, y)}{\partial x}\frac{df_{1}}{dt} + \frac{\partial \psi(x, y)}{\partial y}\frac{df_{2}}{dt}
        \]
        we can replace $t$ with $x$, let $f_{1} \equiv x$ and $f_{2} \equiv y(x)$, then
        \begin{alignat}{1}
            \frac{d\psi(f_{1}(x), f_{2}(x))}{dx} &= \frac{\partial \psi(x, y)}{\partial x}\frac{df_{1}}{dx} + \frac{\partial \psi(x, y)}{\partial y}\frac{df_{2}}{dx}\\
            &= \frac{\partial \psi(x, y)}{\partial x} + \frac{\partial \psi(x, y)}{\partial y}\frac{dy}{dx}
        \end{alignat}
        finally,
        \[
            N(x, y)\frac{dy}{dx} = \frac{\partial \psi(x, y)}{\partial y} \frac{dy}{dx} = \frac{d\psi(x, y(x))}{dx} - \frac{\partial \psi(x, y)}{\partial x}
        \]
    \end{remark}
    \begin{remark}
        Notice in \textbf{Equation \ref{6232}} has 3 variables: $h, x, y$; however, the terms with $x$ cancel, leaving just $h$ and $y$. This occurs due to the equation being exact.
    \end{remark}
\end{document}