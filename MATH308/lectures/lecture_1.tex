% !TeX root = ../diffeq.tex
\documentclass[diffeq.tex]{subfiles}

\begin{document}
\chapter{14 January 2020}
    \section{First Order Differential Equations}
    \begin{definition}[Differential Equation]
        A \textbf{differential equation} (DE) is an equation that relates a function to its derivative, for example
        \begin{equation}
            \frac{dy}{dx} = f(x)
        \end{equation}
        For our class, we will only consider \textbf{ordinary differential equations} (ODE). An ODE relates one independent variable to its derivatives, where as something like \textbf{partial differential equations} relates multiple independent varibles to their derivatives.
    \end{definition}
    \begin{definition}[First Order Differential Equation]
        \textbf{First order differential equations} are a class of differential equations and are of the form
        \begin{equation}
            \frac{dy}{dt} = f(t, y)
        \end{equation}
        A function $y = \phi(t)$ is considered to be the solution to the differential equation iff $y$ is differentiable for all $t$ in some interval.
        For this class we will generally use $t$ as opposed to $x$ as the independent variable, as most differential equations model changes with respect to time.
    \end{definition}
    \begin{definition}[General Solutions]
        A solution to a differential equation is considered \textbf{general} if there exists arbitrary constants in the answer. It is usually denoted as $y_{c}$
        \begin{equation}
            y_{c} = c_1f_{1}(t) + c_2f_{2}(t) + \dots + c_nf_{n}(t)
        \end{equation}
    \end{definition}
    \np
    \begin{example}
        \label{ex1-1}
        Find the general solution to the following
        \begin{equation}
            y' = 1
        \end{equation}
        Remember, the solution to a differential equation when differentiated must satisfy the original equation.\\[1em]
        \noindent
        \textbf{Solution:}\\
        First, we can integrate both sides and solve for $y_c$
        \begin{alignat}{1}
            \int y'\,dt &= \int 1\,dt\\
            \implies y_{c}(t) &= t + c
        \end{alignat}
    \end{example}
    \begin{definition}[Open Differential Equations]
        There are many \textbf{open differential equations}, meaning that there do not exist a trivial solution to them.
    \end{definition}
    \begin{example}[Open Differential Equation]
        Given the following, solve for the general solution.
        \begin{equation}
            y' = x'y - x^{3}
        \end{equation}
        \textbf{Solution:}\\
        This is an \textbf{open} differential equation. There is not a method which solves this easily.
    \end{example}
    \begin{example}
        \label{ex1-3}
        Given the following, solve for the general solution.
        \begin{equation}
            y' = y
        \end{equation}
        \textbf{Solution:}\\
        Recall back to previous experience in calculus,
        \begin{equation}
            \int e^{t}\,dt = e^{t}
        \end{equation}
        Then it becomes trivial to solve this differential equation,
        \begin{equation}
            y' = e^t \implies y \int y'\,dt = \int e^t\,dt
        \end{equation}
        Using the above, the general solution can be found
        \begin{equation}
            y_c(t) = ce^{x}
        \end{equation}
    \end{example}
    \np
    \begin{remark}[Regarding Example \ref{ex1-3}]
        If both sides of a differential equation are dependent ont he same variable --- i.e. the same variable appears on both sides of the equation --- then integrating both sides cannot be used to solve the equation as we did in \textbf{Example \ref{ex1-1}}.
    \end{remark}
    \begin{definition}[Initial Value Problem]
        An \textbf{initial value problem} (IVP), otherwise known as an \textbf{initial condition problem}, is a problem where the solution of a differential equation is dependent on its initial conditions. This then leads to a \textbf{unique solution}.
    \end{definition}
    \begin{example}[Initial Value Problem]
        Solve
        \begin{equation}
            y' = x,\ y(0) = 1
        \end{equation}
        \textbf{Solution:}\\
        Notice that this is an IVP as an initial condition, $y(0) = 1$, is given.\\
        First, solve for the general solution
        \begin{alignat}{1}
            \int y'\,dx &= \int x\,dx\\
            \implies y_c &= \frac{1}{2}x^{2} + c
        \end{alignat}
        Then use the initial value to solve for $c$
        \begin{alignat}{1}
            y_c(0) &= 0 + c = 1\\
            \implies c &= 1
        \end{alignat}
        Finally, plug in $c$ to $y_{c}$ to get the unique solution
        \begin{equation}
            y = \frac{1}{2}x^{2} + 1
        \end{equation}
    \end{example}
    \np
    \section{Differentiable Functions}
    \begin{definition}[Differentiability]
        A function is \textbf{differentiable} for some value $t \in \RE$ iff
        \begin{equation}
            \exists f'(t) = \lim_{h\to0}\frac{f(t + h) - f(t)}{h}
        \end{equation}
        Or more graphically,
        \begin{center}
            \begin{tikzpicture}
                \draw[<->] (-1,0) -- (5,0) node[right] {$t$};
                \draw[<->] (0,-1) -- (0,5) node[above] {$y$};
                \draw[red,<->] (-1,1.625) --(1,1.625);
                \node[red,label={[xshift=5]A}] at (0,1.6) {\textbullet};
                \node[black,label={B}] at (2.6,.585) {\textbullet};
                \draw[scale = .65, domain=-1:4,smooth,variable=\x,blue] plot ({\x}, {-.1*\x*\x+2.5});
                \draw[scale = .65, domain=4:8,smooth,variable=\x,blue] plot ({\x}, {-.1*(\x-14)*(\x-14)+10.9});
            \end{tikzpicture}
        \end{center}
        A function is considered to be \textbf{differentiable} at some point $t$ if there exists only one tangent line at at $t$. In the picture at point A, there is only one tangent line, however, at point B there exists an infinite number of tangent lines.
    \end{definition}
    \np
    \section{Applications of Differential Equations}
    ODEs and DEs in general have many real life applications, for instance they can be used to model predator-prey populations, or the filling of a tank (future lecture), even heat transfer (PDEs).
    \begin{example}[Kinematics]
        For our first application, we will consider the simple case of kinematics.\\
        Given an object travelling at a velocity $v_{0}$ with constant acceleration of $a$, find the position $s$ at any time $t$.
        \textbf{Solution:}
        From prior calculus or physics knowledge, we know the rate of change of velocity is equal to acceleration
        \begin{equation}
            \frac{d}{dt}v = a
        \end{equation}
        Notice that this is a differential equation, and we can integrate both sides
        \begin{alignat}{1}
            \int \frac{d}{dt}v\,dt &= \int a\,dt\\
            \implies v &= at + c
        \end{alignat}
        Also notice that this is an IVP, so we can solve for the constant
        \begin{equation}
            v(0) = 0 + c = v_{0} \implies c = v_{0}
        \end{equation}
        Now, recall from prior knowledge the relationship between position and velocity
        \begin{equation}
            \frac{d}{dt}s = v\,dt
        \end{equation}
        Solving this differential equation yields
        \begin{equation}
            s = \int v\,dt
        \end{equation}
        Substituting in our solution to $v$ gives us
        \begin{equation}
            s = \int(at + v_{0})\,dt
        \end{equation}
        If we also assume that $s(0) = 0$, we can reach the following solution
        \begin{equation}
            s = \frac{1}{2}at^{2} + v_{0}t
        \end{equation}
    \end{example}
\end{document}