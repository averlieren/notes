% !TeX root = ../diffeq.tex
\documentclass[diffeq.tex]{subfiles}

\begin{document}
\chapter{16 January 2020}
    \section{Linear Differential Equations}
    \begin{definition}[First Order Linear Differential Equations]
        \begin{equation}
            \underbrace{y' + p(t)y = g(t)}_\text{\tiny Usual form}\iff y' = g(t) - p(t)y
        \end{equation}
        A \textbf{first order linear differential equation} (LDE) is linear due to $y$ being dependent on only one variable, $t$.
    \end{definition}
    Notice that $t$ is typically used in place of $x$ as most differential equations are used in models dependent on time; as such, most differential equations are in the form $y' = f(t, y)$ as opposed to $y' = f(x, y)$.
    \begin{example}
        \begin{alignat}{3}
            &\text{Solve } &(4 + t^{2})y' + 2ty &= 4t\\
            &\text{Notice: }&(4y + t^{2}y)' &= \frac{d}{dt}(4y + t^{2}y) = 4t\\
            &&&= 4y' + (t^{2}y)^{2}\\
            &&&= 4y' + (2ty + t^{2}y')\\
            &&&= (4 + t^2)y' + 2ty
        \end{alignat}
        The original problem can now be reduced to:
        \begin{alignat}{3}
            &&\frac{d}{dt}(4y + t^2y)&= 4t\\
            &&\text{let\quad}z(t) &= 4y + t^{2}y\\
            &&&= 2 t^{2} + C\\
            &\implies& 4y + t^{2}y &= 2 t^{2}+C\\
            &\therefore& y &= \frac{1}{4+t^{2}}(2 t^{2}+C)
        \end{alignat}
    \end{example}
    \begin{remark}[Constants]
        Notice in the above example that the constant, $C$, is being multiplied by $\frac{1}{4 + t^{2}}$. When expanding the answer, it now becomes $y = \frac{2t^{2}}{4+t^{2}} + \frac{C}{4+t^{2}}$. Notice how the constant is dependent on the variable $t$, and is therefore not the same as just $C$.
    \end{remark}
    \begin{definition}[Integrating Factors with LDEs]
        An \textbf{integrating factor}, $\mu(t)$ is a function $\mu(t): \RR \to \RR$, that satisfies $\frac{d}{dt}\mu(t) = \mu(t)y'+\mu(t)p(t)y$.
    \end{definition}
    \begin{remark}
        There are infinitely many integrating factors due to the arbitrary constant $C$ from indefinite integration, see \textbf{Method 2.1} and \textbf{Example 2.2} on the following page.
    \end{remark}
    \np
    \begin{method}[Solution of the General LDE Case]
        Solve $y' + p(t)y = g(t)$.
        \begin{enumerate}
            \item Multiply the LDE by $\mu(t)$ results in:
                \begin{equation}
                    \mu(t)(y' + p(t)y) = \mu(t)g(t)
                \end{equation}
            \item Letting $z(t) = \mu(t)y$, and $z' = \mu(t)g(t)$ yields:
                \begin{align}
                    z(t) &= \int \mu(t)g(t)\,dt\\
                    \implies y(t) &= \frac{1}{\mu(t)}\int \mu(t)g(t)\,dt\\
                    \implies \mu(t) &= \exp\BBp{\int p(t)\,dt}
                \end{align}
            \item Therefore the solution of the general case is
                \begin{equation}
                    y(t) = \BBp{\exp\Bp{\int p(t)\,dt}}^{-1}\cdot \int \exp\Bp{\int p(t)\,dt}g(t)\,dt
                \end{equation}
        \end{enumerate}
    \end{method}
    \begin{example}[Solving an IVP involving LDEs]
        Working with example 2.1.4 from the textbook:
        \begin{equation}
            ty' +4 2y = 4 t^{2},\ y(1) = 2
        \end{equation}
        \begin{enumerate}
            \item Compute the integrating factor ($\mu(t)$)
                \begin{align}
                    \mu(t) &= \exp\bbbp{\int p(t)\,dt}\\
                    &= \exp\bbbp{\int 2 t^{-1}\,dt}\\
                    &= \exp\bp{2\ln(t) + C} \Leftrightarrow e^{2\ln(t) + C}
                \end{align}
            \item Find the general case\\
                When solving, $0$ can be subsituted in for $C$ to simplify calculations; for $C \neq 0$ it is trivially shown that the constant will cancel out in computing the solution.
                \begin{align}
                    y_c(t) &= \frac{1}{\mu(t)}\int \mu(t)g(t)\,dt\\
                    &= \frac{1}{t^{2}}\bbbp{\int t^{2}\cdot 4t\,dt}\\
                    &= \frac{1}{t^{2}}(t^{4} + C)
                \end{align}
                \textbf{Note:} $y_c(t)$ is used to denote the general case.
            \item Find formula w.r.t. intial value
                \begin{alignat}{3}
                    &&y(1) &= 2\\
                    &\implies &y(1) &= (1)^{2} + \frac{C}{(1)^{2}}\\
                    &\implies &C &= 1\\
                    &\therefore &y(t) &= t^{2} + t^{-2}
                \end{alignat}
        \end{enumerate}
    \end{example}
\end{document}