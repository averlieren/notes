% !TeX root = ../diffeq.tex
\documentclass[diffeq.tex]{subfiles}

\begin{document}
\chapter{26 March 2020}
    \section{Inverse Laplace Transform}
    \begin{definition}
        If
        \begin{equation}
            F(s) = \lp{f(t)}
        \end{equation}
        then, the \textbf{inverse Laplace transform} is
        \begin{equation}
            f(t) = \lpinv{F(s)}
        \end{equation}
    \end{definition}
    Addendum to the previous lecture, when solving for:
    \begin{equation}
        y'' + y = \sin(2t), y(0) = 2, y'(0) = 1
    \end{equation}
    We came to the point,
    \begin{equation}
        \lp{y} = \frac{2s + 1}{s^{2} + 1}\cdot\frac{2}{(s^{2} + 1)(s^{2} + 4)}
    \end{equation}
    So then,
    \begin{equation}
        y = \lpinv{\frac{2s + 1}{s^{2} + 1}\cdot\frac{2}{(s^{2} + 1)(s^{2} + 4)}}
    \end{equation}
    \subsection{Step Functions} % TODO: Add graphs
    For every function $f: \RR \to \RR$, there exists a graph. Most graphs we are accustomed to seeing are continuous. However, there exists step functions where there are a finite number of steps.
    \np
    \begin{definition}
        The unit step function can be defined as follows:
        \begin{equation}
            u_{c}(t) = \begin{cases}
                0 & t < c\\
                1 & t \geq c
            \end{cases}
        \end{equation}
    \end{definition}
    \begin{example}
        Consider
        \begin{equation}
            f(t) = \begin{cases}
                2 & t \in [0, 4) \\
                5 & t \in [4, 7) \\
                -1 & t \in [7, 9) \\
                1 & t \geq 9
            \end{cases}
        \end{equation}
        % TODO: Insert graph
        It can be written as the linear combination of multiple unit step functions:
        \begin{equation}
            f(t) = 2u_{0}(t) + 3u_{4}(t) - 6u_{7}(t) + 2u_{9}(t)
        \end{equation}
    \end{example}
    \begin{btheorem}
        \begin{equation}
            \lp{u_{c}(t)\cdot f(t-c)} = e^{-cs}\lp{f(t)}, t > c
        \end{equation}
    \end{btheorem}
    \begin{corollary}
        \begin{equation}
            \lpinv{e^{-cs}F(s)} = u_{c}(t)f(t-c)
        \end{equation}
    \end{corollary}
    \begin{example}
        Compute the following
        \begin{equation}
            \lpinv{e^{-2s}\cdot \frac{1}{s}}
        \end{equation}
        Note that
        \begin{equation}
            \lpinv{\frac{1}{s}} = 1
        \end{equation}
        \begin{equation}
            \lpinv{e^{-2s}\cdot \frac{1}{s}} = u_{c}(t)\cdot f(t-2)
        \end{equation}
    \end{example}
    \begin{btheorem}
        \begin{equation}
            \lp{e^{ct}f(t)} = F(s - c)
        \end{equation}
        \begin{equation}
            \lpinv{F(s)} = e^{ct}\lpinv{F(s + c)}
        \end{equation}
    \end{btheorem}\np
    Also, in the last lecture
    \begin{equation}
        \lp{y} = \frac{\frac{1}{3}}{s - 2} + \frac{\frac{2}{3}}{s + 1}
    \end{equation}
    Notice that in both fractions, it can be expressed as a linear combination of the following function
    \begin{equation}
        F(s) = \frac{1}{s}
    \end{equation}
    rewriting the expression yields:
    \begin{equation}
        \lp{y} = \frac{1}{3}F(s - 2) + \frac{2}{3}F(s + 1)
    \end{equation}
    Now solving for y
    \begin{equation}
        y = \lc^{-1}\left\{\frac{1}{3}F(s - 2) + \frac{2}{3}F(s + 1)\right\}
    \end{equation}
    The inverse Laplacian can also be written as a linear combination
    \begin{equation}
        y = \lc^{-1}\left\{\frac{1}{3}F(s - 2)\right\} + \lc^{-1}\left\{\frac{2}{3}F(s + 1)\right\}
    \end{equation}
    Remember to \textbf{Theorem 17.2}, we can adjust $F(s + c)$ back to $F(s)$:
    \begin{equation}
        y = \frac{1}{3}e^{2t}\lpinv{F(s + 2)} + \frac{2}{3}e^{t}\lpinv{F(s + 1)}
    \end{equation}
    Now, notice that $\lpinv{F(s)} = 1$, and rearranging \textbf{Theorem 17.2}, we find that $\lpinv{F(s + c)} = e^{-ct}\lpinv{F(s)}$. If we apply this to our solution, we get the final solution:
    \begin{equation}
        y = \frac{1}{3}e^{-2t}+\frac{2}{3}e^{-t}
    \end{equation}
\end{document}