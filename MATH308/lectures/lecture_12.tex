% !TeX root = ../diffeq.tex
\documentclass[diffeq.tex]{subfiles}

\begin{document}
\chapter{20 February 2020}
\section{Imaginary Roots}
Now consider a characteristic function,
\begin{equation}
    f(r) = r^{2} + ar + b = 0
\end{equation}
where $r \in \CC$. The roots of $f(r) = 0$ can then be expressed by
\begin{equation}
    r_{1,2} = \lambda \pm i\tau
\end{equation}
Then, what is the general solution of
\begin{equation}
    L[y] = y'' + ay' + by = 0
\end{equation}
where the characteristic function yields non-real roots?
\begin{btheorem}[Imaginary Roots]
    In the case that the solutions to $f(r) = 0$ are
    \begin{equation}
        r_{1,2} = \lambda \pm i\tau
    \end{equation}
    the general solution of the corresponding $L[y] = 0$ is given by
    \begin{equation}
        y_{1} = e^{\lambda t}\cos(\tau t);\quad y_{2} = e^{\lambda t}\sin(\tau t)
    \end{equation}
    
    The solutions $y_{1}$ and $y_{2}$ are a fundamental set of solutions.
\end{btheorem}
\begin{homework}
    Show that $y_{1} = e^{\lambda t}\cos(\tau t);\quad y_{2} = e^{\lambda t}\sin(\tau t)$ are solutions to $L[y] = 0$, with a characteristic function that has imaginary roots.
\end{homework}
\begin{bproof}[Fundamental Set]
    In \textbf{Theorem 12.1}, $y_{1} = e^{\lambda t}\cos(\tau t);\quad y_{2} = e^{\lambda t}\sin(\tau t)$
    \begin{alignat}{1}
        &W[y_{1}, y_{2}]\\
        &= \begin{vmatrix}
            e^{\lambda t}\cos(\tau t) & e^{\lambda t}\sin(\tau t)\\
            \lambda e^{\lambda t}\cos(\tau t) - \tau e^{\lambda t}\sin(\tau t) & \lambda e^{\lambda t}\sin(\tau t) + \tau e^{\lambda t}\cos(\tau t)
        \end{vmatrix}\\
        &= e^{2\lambda t}\begin{vmatrix}
            \cos(\tau t) & \sin(\tau t)\\
            \lambda \cos(\tau t) - \tau\sin(\tau t) & \lambda\sin(\tau t) + \tau\cos(\tau t)
        \end{vmatrix}\\
        &= e^{2\lambda t}[\lambda\cos(\tau t)\sin(\tau t) + \tau\cos^{2}(\tau t) - \lambda\sin(\tau t)\cos(\tau t) + \tau\sin^{2}(\tau t)]\\
        &= \tau e^{2\lambda t} \nequiv 0
    \end{alignat}
    Therefore, $y_{1,2}$ form a fundamental set of solutions.
\end{bproof}
\begin{example}
    Solve
    \begin{equation}
        L[y] = y'' - 2y' + 6y = 0
    \end{equation}
    The characteristic function is given by
    \begin{equation}
        f(r) = r^{2} - 2r + 6 = 0
    \end{equation}
    The roots are
    \begin{equation}
        r_{1,2} = \frac{2 \pm \sqrt{-20}}{2} = 1 \pm i\sqrt{5}
    \end{equation}
    Therefore,
    \begin{equation}
        y_{1} = e^{t}\cos\left(\sqrt{5}t\right);\quad y_{2} = e^{t}\sin(\sqrt{5}t)
    \end{equation}
    Then, the general solution can be given as
    \begin{equation}
        y_{c} = c_{1}e^{t}\cos(\sqrt{5}t) + c_{2}e^{t}\sin(\sqrt{5}t)
    \end{equation}
\end{example}
\np
\begin{example}
    Solve
    \begin{equation}
        L[y] = 9y'' + 6y' + y = 0
    \end{equation}
    The characteristic function is given by
    \begin{equation}
        f(r) = r^{2} + 6y' + y = 0
    \end{equation}
    The roots are
    \begin{equation}
        (3r + 1)^{2}
    \end{equation}
    \begin{equation}
        r_{1} = r_{2} = -\frac{1}{3}
    \end{equation}
    This implies
    \begin{equation}
        y_{1} = \exp(-\frac{1}{3}t);\quad y_{2} = t\exp(-\frac{1}{3}t)
    \end{equation}
    Then, the general solution can be given as
    \begin{equation}
        y_{c} = c_{1}\exp(-\frac{1}{3}t) + c_{2}t\exp(-\frac{1}{3}t)
    \end{equation}
\end{example}
\end{document}