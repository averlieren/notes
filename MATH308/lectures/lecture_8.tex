% !TeX root = ../diffeq.tex
\documentclass[diffeq.tex]{subfiles}

\begin{document}

\chapter{6 February 2020}
\section{First Order Differential Equation Review}
Topics covered in First Order Differential Equations.
\begin{itemize}
    \item $y' = f(x, y)$\quad$y(x_{0}) = y_{0}$
    \item First Order LDE, $y' + p(t)y = g(t)$
    \item Separable, $y' = \frac{M(x, y)}{N(x, y)}$
    \item Exact, $M(x, y) + N(x, y)y' = 0$; $M_{y}(x, y) = N_{x}(x, y)$
    \item Uniqueness and Existance Theorems
    \item Modelling
\end{itemize}
\section{Second Order Differential Equations}
\begin{definition}[Second Order Differential Equations]
    The general form of a \textbf{second order differential equation} (SODE) is
    \begin{equation}
        y'' = f(x, y, y');\quad y(x_{0}) = y_{0};\quad y'(x_{0}) = y_{1}
    \end{equation}
\end{definition}
\np
\begin{example}
    The following are SODEs,
    \begin{alignat}{1}
        y'' &= 1\\
        y'' &= 1 + y'\\
        y'' &= \frac{x}{t}
    \end{alignat}
    An example of a SODE IVP,
    \begin{equation}
        y'' = x + y + y';\quad y(0) = 1;\quad y'(0) = -3
    \end{equation}
\end{example}
\begin{definition}[Second Order Linear Differential Equations]
    A \textbf{second order linear differential equation} (SOLDE) has the general form
    \begin{equation}
        y'' + p(t)y' + q(t)y = g(t)
    \end{equation}
    where $p(t)$, $q(t)$, $g(t)$ are continuous over some interval $I$.
\end{definition}
\begin{btheorem}[SOLDE Uniqueness Theorem]
    If $p(t), q(t), g(t)$ are continuous in some interval $I: (\alpha, \beta)$\\
    Then, for any $t_{0} \in I$, the IVP defined by
    \begin{equation}
        y'' + p(t)y' + q(t)y = g(t),\quad y(x_{0}) = y_{0},\quad y'(x_{0}) = y_{1}
    \end{equation}
    has a unique solution.
\end{btheorem}
\begin{definition}[Cases of SOLDEs]
    \textbf{Homogeneous SOLDEs} (HSOLDE) are of the following form
    \begin{equation}
        y'' + p(t)y' + q(t)y = 0
    \end{equation}
    If a Homogeneous SODE is defined where $p(t)$, and $q(t)$ are constants, it is considered as a \textbf{homogeneous SOLDE with Constant Coefficients} (CHSOLDE).
\end{definition}
\np
\section{Homogeneous Second Order Linear Differential Equations with Constant Coefficients}
\begin{example}[CHSOLDE]
    Find the general solution of
    \begin{equation}
        \label{CHSOLDE1}
        L[y] = y'' + 5y' + 6y = 0
    \end{equation}
    Consider the following quadratic (characteristic function, or characteristic polynomial).
    \begin{equation}
        f(r) = r^{2} + 5r + 6 = 0
    \end{equation}
    There are 2 different roots to the characteristic function,
    \begin{equation}
        r_{1} = -3;\quad r_{2} = -2
    \end{equation}
    Now consider the equations,
    \begin{equation}
        y_{1}(t) = e^{r_{1}t} = e^{-2t}
    \end{equation}
    \begin{equation}
        y_{2}(t) = e^{r_{1}t} = e^{-3t}
    \end{equation}
    Then, $y_{1}(t)$ and $y_{2}(t)$ are solutions of \textbf{Equation \ref{CHSOLDE1}}.\\
    \textbf{Proof:}
    \begin{alignat}{1}
        &y_{1}'(t) = -3e^{-3t};\quad y_{1}'' = 9e^{-3t}\\
        L[y_{1}] &= 9e^{-3t} + 5(-3)e^{-3t} + 6e^{-3t} = 0 \\
        0&= (9 - 15 + 6)e^{-3t}
    \end{alignat}
    Therefore, the general solution to \textbf{Equation \ref{CHSOLDE1}}
    \begin{alignat}{1}
        y_{c} = C_{1}e^{-3t} + C_{2}e^{-2t}
    \end{alignat}
    where $C_1$ and $C_2$ are constants.
\end{example}
\np
\begin{example}
    \begin{equation}
        L[y] = y'' + ay' + by = 0
    \end{equation}
    \begin{equation}
        f(r) = r^{2} + ar + b = 0
    \end{equation}
    Suppose that $r_{0}$ is a root of $f(r) = 0$\\
    Consider \begin{alignat}{1}
        y_{0}(t) &= e^{r_{0}t}\\
        y_{0}'(t) &= r_{0}e^{r_{0}t}\\
        y_{0}''(t) &= r_{0}^{2}e^{r_{0}t}\\
        \implies L(y_{0}) &= r_{0}^{2}e^{r_{0}t} + ar_{0}e^{r_{0}t} + be^{r_{0}t}\\
        &=e^{r_{0}t}(r_{0}^{2} + ar_{0} + b)
    \end{alignat}
    Things to consider, what if $r_{0} \in \CC$ or $r_{0} = r_{1}$?
\end{example}
\begin{example}[CHSOLDE IVP]
    Find the solution of the CHSOLDE IVP,
    \begin{equation}
        L[y] = y'' + 5y' + 6y = 0;\quad y(0) = 2;\quad y'(0) = 3
    \end{equation}
    \begin{enumerate}
        \item Find the general solution
        \begin{equation}
            y_{c}(t) = C_{1}y_{1} + C_{2}y_{2} \implies y_{c}(t) = C_{1}e^{-3t} + C_{2}e^{-2t}
        \end{equation}
        \item Find the particular values of $C_1$ and $C_2$ such that $C_{1}y_{1}(0) + C_{2}y_{2}(0) = 2$ and $(C_{1}y_{1}(0) + C_{2}y_{2}(0))' = 3$.
        \begin{equation}
            \begin{cases}
                C_{1}e^{-3(0)} + C_{2}e^{-2(0)} = 2\\
                -3C_{1}e^{-3(0)} + -2C_{2}e^{-2(0)} = 3\\
            \end{cases}
        \end{equation}
        Solving the linear combination yields $C_{1} = 7$, $C_{2} = 9$.
        Then, the solution to this IVP is
        \begin{equation}
            y(t) = -7e^{-3t} + 9 e^{-2t}
        \end{equation}
    \end{enumerate}
\end{example}
\end{document}