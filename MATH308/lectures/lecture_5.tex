% !TeX root = ../diffeq.tex
\documentclass[diffeq.tex]{subfiles}

\begin{document}
\chapter{28 January 2020}
    \section{Mathematical Modelling (cont.)}
    \begin{example}[Modelling, 2.3.1]
        \label{ex5-1}
        Find the amount of salt in the tank at time $t$, and find the limiting amount of salt after a long time.
        Given: At time $t = 0$, a tank contains $Q_{0}$ lb of salt dissolved into 100 gal of water. Assume that water containing $\frac{1}{4}$lb of salt per gallon enters the tank at a rate of $r$ gal/min and leaves at the same rate. Assume that the mixture is stirred until it reaches a homogeneous mixture.\\[1em]
        \textbf{Solution:}
        Using the general model derived last lecture
        \begin{equation}
            Q'(t) = I(t)\gamma(t) - \frac{O(t)}{V(t)}Q(t)
        \end{equation}
        \begin{alignat}{1}
            I(t) &= r\ \text{gal}\,\text{min}^{-1}\\
            O(t) &= r\ \text{gal}\,\text{min}^{-1}\\
            V(t) &= 100\ \text{gal}\\
            \gamma(t) &= \frac{1}{4}
        \end{alignat}
        \begin{alignat}{1}
            Q' &= \frac{1}{4}r - \frac{r}{100}Q\\
            Q' + \frac{r}{100}Q &= \frac{1}{4}r\\
            \mu(t) &= e^{\frac{r}{100}t}\\
            Q &= e^{-\frac{r}{100}t}\int \left(e^{\frac{r}{100}t}\right)\left(\frac{1}{4}r\right)\,dt\\
            &= e^{-\frac{r}{100}t}\left(\frac{100}{r}e^{\frac{r}{100}t} + c\right)\\
            &= 25 + ce^{-\frac{r}{100}t}
        \end{alignat}
        Plugging in $Q(0) = Q_{0}$
        \begin{equation}
            c = (Q_{0} - 25)e^{-\frac{r}{100}t}
        \end{equation}
        Then the solution is given by
        \begin{equation}
            Q(t) = 25 + (Q_{0} - 25)e^{-\frac{r}{100}t}
        \end{equation}
        After a long time,
        \begin{equation}
            \lim_{t\to\infty} Q(t) = 25
        \end{equation}
    \end{example}
    \begin{remark}[Regarding Example \ref{ex5-1}]
        Notice that after a long time, regardless of the inital amount of salt, $Q_{0}$, the tank will always end up with 25 lbs of salt.
    \end{remark}
    \np
    \section{Exact Differential Equations}
    \begin{definition}[Exact DE]
        Given the following ODE
        \begin{equation}
            M(x, y)\,dx + N(x, y)\,dy = 0
        \end{equation}
        It is considered to be exact iff
        \begin{equation}
            \frac{\partial M(x,y)}{\partial y} = \frac{\partial N(x,y)}{\partial x} \Leftrightarrow N(x, y)y' + M(x, y) = 0
        \end{equation}
    \end{definition}
    (To be honest, I don't know what's going on here, it was on the board so I copied it down.)\\[1em]
    Given a function $\psi(x, y)$, we can define a parametric function $\delta(t) = \psi(f_{1}(t), f_{2}(t))$, where $x(t) = f_{1}(t)$, and $y(t) = f_{2}(t)$. Therefore, $\psi = \delta$
    \begin{alignat}{1}
        \frac{d}{dt}\psi(x, y) &= \frac{d}{dt}\delta\\
        &= \frac{\partial \psi(x, y)}{\partial x}\frac{df_{1}}{dt} + \frac{\partial \psi(x, y)}{\partial y}\frac{df_{2}}{dt}
    \end{alignat}
    \begin{example}
        \begin{alignat}{1}
            \psi(x, y) &= x^{2}y + xy\\
            f_{1}(t) = t&,\ f_{2}(t) = t^{2}\\
            \delta(t) &= \psi(f_{1}, f_{2})\\
            &= t^{2}t^{2}+t t^{2}\\
            \delta'(t) &= 4t^{3} + 3t^{2}\label{ex1-2-5}\\
            \frac{\partial \psi(x, y)}{\partial x}\cdot 1 &+ \frac{\partial \psi(x, y)}{\partial y}\cdot2t\\
            &= (2f_{1}f_{2} + f_{2})\cdot 1 + (f_{1}^{2} + f_{1})\cdot 2ty\\
            &= 4t^{3} + 3t^{2}\label{ex1-2-9}
        \end{alignat}
        Notice how \textbf{Equation \ref{ex1-2-5}} and \textbf{Equation \ref{ex1-2-9}} are the same, but derived via different methods.
    \end{example}
    \np
    \begin{example}
        Show that the following equation is exact.
        \begin{equation}
            y' = \frac{\frac{1}{y}}{x}
        \end{equation}
        \textbf{Solution:}
        \begin{equation}
            M(x, y) = \frac{1}{x},\ N(x, y) = y
        \end{equation}
        \begin{alignat}{2}
            \frac{\partial}{\partial y}M(x, y) &= \frac{\partial}{\partial y}x^{-1} &= 0\\
            \frac{\partial}{\partial x}N(x, y) &= \frac{\partial}{\partial x}y &= 0
        \end{alignat}
        Therefore, the equation is exact.
    \end{example}
    \begin{theorem}[Exactness]
        If the following equation is exact
        \begin{equation}
            M(x, y)\,dx + N(x, y)\,dy = 0
        \end{equation}
        then, $\exists\,\psi(x, y)$ s.t.
        \begin{equation}
            \frac{\partial}{\partial x}\psi(x, y) = M(x, y),\ \frac{\partial}{\partial y}\psi(x, y) = N(x, y)
        \end{equation}
    \end{theorem}
    \begin{remark}[Separable and Exact]
        You may notice that exact ODEs look very similar to separable ODEs. That is because exact ODEs are a superset of separable ODEs, i.e. all separable ODEs are exact.
    \end{remark}
\end{document}