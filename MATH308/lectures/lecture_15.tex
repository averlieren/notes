% !TeX root = ../diffeq.tex
\documentclass[diffeq.tex]{subfiles}

\begin{document}
\chapter{3 March 2020}
    \section{Indefinite Integrals}

    \begin{definition}[Improper Integrals]
        \begin{equation}
            \int_{a}^{\infty}f(t)\,dt = \lim_{A\to\infty}\int_{a}^{A}f(t)\,dt
        \end{equation}
        If the limit exists then it is convergent, otherwise it is considered to be divergent
    \end{definition}
    \np
    \begin{example}
        Does the following integral converge?
        \begin{equation}
            \int_{1}^{\infty} t^{-1}\,dt
        \end{equation}
        Solution:
        \begin{alignat}{1}
            \lim_{A\to\infty}\int_{1}^{A}(t^{-1})\,dt&\\
            \lim_{A\to\infty}\left(\ln(t)|_{1}^{A}\right) &= \infty
        \end{alignat}
        Therefore, it is divergent.
    \end{example}
    \begin{example}
        For what values does the following integral converge?
        \begin{alignat}{1}
            &\int_{0}^{\infty}e^{ct}\,dt\\
            =&\lim_{A\to\infty}\int_{0}^{A}e^{ct}\,dt\\
            =&\lim_{A\to\infty}\frac{1}{c}e^{ct}\bigg|_{0}^{A}\\
            =&\lim_{A\to\infty}\left[\frac{1}{c}e^{ct}-\frac{1}{c}\right]\\
            =& \frac{1}{c}, c < 0
        \end{alignat}
    \end{example}
    \np
    \begin{example}
        For what values of $p$ does the integral converge?
        \begin{alignat}{1}
            &\int_{1}^{\infty}t^{-p}\,dt\\
            =&\lim_{A\to\infty}\int_{1}^{A}t^{-p}\,dt\\
            =&\lim_{A\to\infty}\frac{1}{1-p}t^{1-p}\bigg|_{1}^{A}\\
            =&\lim_{A\to\infty}\frac{1}{1-p}\left(A^{1-p}-1\right), p \neq -1\\
            =& \frac{1}{p - 1}, p > 1
        \end{alignat}
    \end{example}
    % TODO: Write stuff about piecewise continuity here, was not covered in original lecture notes
    \section{Laplace Transformation}
    In algebra, we were introduced to the concept of factorization,
    \begin{alignat}{1}
        x^{2} + 4x + 3 &= 0 \\
        (x + 3)(x + 1) &= 0 \\
        \implies x \in \{-2, -1\}
    \end{alignat}
    Factorization is useful because it was a tool to solve for the roots of a polynomial. The Laplace tranformation can also be thought of as a tool that would help in solving ODEs.

    \begin{definition}
        The \textbf{Laplace Transformation} is an integral transformation, given by
        \begin{equation}
            \lp{f(t)} = F(s) = \int_{0}^{\infty}e^{-st}f(t)\,dt
        \end{equation}
    \end{definition}

    \begin{theorem} % TODO: Add proof
        Suppose that
        \begin{enumerate}
            \item $f$ is piecewise continuous on the intervals $t\in[0,A], A \in \RE^{+}$
            \item $\exists (k, a, M), (K, M) > 0, |f(t)| \leq ke^{at}, t \geq M$
        \end{enumerate}
        Then, $\forall s > a, F(s)$
    \end{theorem}
    \np
    \begin{example} % TODO: workout this example
        Find \begin{equation}
            \lp{1} = \frac{1}{s^{2}}
        \end{equation}
    \end{example}
    \chapter*{5 - 19 March 2020}
    Lectures on dates 5, 17, 19 March were cancelled.
\end{document}