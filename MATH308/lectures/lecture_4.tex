% !TeX root = ../diffeq.tex
\documentclass[diffeq.tex]{subfiles}

\begin{document}
\chapter{23 January 2020}
    \section{Separable Equations (cont.)}
    \begin{example}[Textbook 2.2.2]
        Solve and then determine the interval in which the solution exists.
        \begin{equation}
            \frac{dy}{dx} = \frac{3x^{2} + 4x + 2}{2(y - 1)},\ y(0) = -1
        \end{equation}
        \textbf{Solution:}
        \begin{align}
            \int 2(y - 1)\,dy &= \int (3x^{2} + 4x + 2)\,dx\\
            \implies y^{2} - 2y &= x^{3} + 2x^{2} + 2x + c
        \end{align}
        Solving for $c$ by plugging in $y(0) = -1$
        \begin{alignat}{1}
            (-1)^{2} - 2(-1) &= 0 + 0 + 0 + c\\
            c &= 3
        \end{alignat}
        Therefore, a unique answer is
        \begin{equation}
            y^{2} - 2y = x^{3} + 2x^{2} + 2x + 3
        \end{equation}
        Rearranging to get an explicit form
        \begin{equation}
            y = \pm \sqrt{x^{3} + 2x^{2} + 2x + 4}
        \end{equation}
        However, only the negative form satisfies the inital condition. If we graphed the function we'll see $\forall x > -2, y'(x)$, which gives us the interval $(-2, \infty)$.
        \begin{equation}
            y = -\sqrt{x^{3} + 2x^{2} + 2x + 4}, x > -2
        \end{equation}
        We can also do an algebraic method to determine this interval. Beginning from our answer above, notice the radical, remember that it is only defined whenever the radicand is non-negative. So we have to solve for the following inequality
        \begin{alignat}{1}
            x^{3} + 2x^{2} + 2x + 4 &\geq -2\\
            (x^{2} + 2)(x + 2) &\geq 0\\
            x &\geq -2
        \end{alignat}
        Therefore, the interval in which the radical is defined is $[-2, \infty)$, but remember that the answer must be differentiable $\forall t \in I$. At $x = -2, y'(x) = \infty$, therefore the derivative does not exist at that point. Now the interval in which the solution exists can be refined to $(-2, \infty)$.
    \end{example}
    \section{Mathematical Modelling}
    \begin{example}[Modelling]
        \label{ex4-2}
        Consider a pond filled with 10 million gallons of fresh water. A flow of 5 million gallons per year with water that is contaminated with a chemical enters the pond. There is also an outflow of this mixture on the order of 5 million gallons per year.\\[1em]
        Let $\gamma(t)$ be the concentration of the fluid entering the pond at time $t$, and let $Q(t)$ be the quantity of chemicals in the pond at time $t$.\\[1em]
        It is determined that
        \begin{equation}
            \gamma(t) = 2 + \sin(2t)\ \text{g}\cdot\text{gal}^{-1}
        \end{equation}
        Find $Q(t)$ using the given information.\\[1em]
        \textbf{Solution:}
        \begin{enumerate}
            \item We can infer that $Q(0) = 0$ because the water starts off fresh.
            \item We know that $Q'$ is the rate of change of the chemicals in the water. Therefore, $Q'$ is equal to the rate at which they enter minus the rate at which they leave,
            \begin{equation}
                Q'(t) = I(t)\gamma(t) - \frac{O(t)}{V(t)}Q(t)
            \end{equation}
            Where $I(t)$ is the rate at which water enters, $\gamma(t)$ is the concentration of chemicals in the entering water, $O(t)$ is the outflow rate, $V(t)$ is the total volume of the system (pond), and $Q(t)$ is the the quantity of chemicals.
        \end{enumerate}
        In this case,
        \begin{alignat}{1}
            I(t) &= 5\times 10^{6}\ \text{gal}\,\text{year}^{-1}\\
            O(t) &= 5\times 10^{6}\ \text{gal}\,\text{year}^{-1}\\
            V(t) &= 10^{7}\ \text{gal}\\
            \gamma(t) &= 2 + \sin(2t)
        \end{alignat}
        Plugging in the values into $Q'(t)$
        \begin{equation}
            Q' = 5\times 10^{6}(2 + \sin(2t)) - \frac{5\times 10^{6}}{10^{7}}Q(t)
        \end{equation}
        The above should look familar as a first order linear ODE.
        \begin{equation}
            Q' + \frac{1}{2}Q(t) = 5 \times 10^{6}(2 + \sin(2t))
        \end{equation}
        Now, we can solve it using our prior knowledge
        \begin{equation}
            \mu(t) = e^{\frac{1}{2}t}
        \end{equation}
        \begin{alignat}{1}
            Q &= 5 \times 10^{6}e^{-\frac{1}{2}t}\int e^{\frac{1}{2}t}(2 + \sin(2t))\,dt\\
            &= 2 \times 10^{7} + \frac{2 \times 10^{7}}{17}\sin(2t) - \frac{4 \times 10^{7}}{17}\cos(2t) + ce^{-\frac{1}{2}t}
        \end{alignat}
        Plugging in $Q(0) = 0$ yields
        \begin{alignat}{1}
            Q(0) &= 2 \times 10^{7} - \frac{4 \times 10^{7}}{17} + c = 0\\
            \implies c &= \frac{-3 \times 10^{8}}{17}
        \end{alignat}
        Therefore, the final answer
        \begin{equation}
            Q(t) = 2 \times 10^{7} + \frac{2 \times 10^{7}}{17}\sin(2t) - \frac{4 \times 10^{7}}{17}\cos(2t) - \frac{3 \times 10^{8}}{17}e^{-\frac{1}{2}t}
        \end{equation}
    \end{example}
    \begin{remark}[Regarding Example \ref{ex4-2}]
        Notice that the solution in the above modelling example starts similarily to exponential growth, but eventually becomes periodic after a certain time.
    \end{remark}
\end{document}