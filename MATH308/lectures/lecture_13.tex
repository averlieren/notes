% !TeX root = ../diffeq.tex
\documentclass[diffeq.tex]{subfiles}

\begin{document}
\chapter{25 February 2020}
    Note: Lectures from 25 February to 3 March, 2020 are not presented by Dr. Darbinyan
    \section{Nonhomogeneous SOLDEs}
    \begin{definition}[Nonhomogeneous SOLDE]
        The general form of a Nonhomogeneous SOLDE (NSOLDE) is given by
        \begin{equation}
            L[y] = y'' + p(t)y' + q(t)y = g(t)
        \end{equation}
    \end{definition}
    \begin{theorem}
        Suppose that $y_{1}, y_{2}$ are solutions of
        \begin{equation}
            L[y] = y'' + p(t)y' + q(t)y = g(t)
        \end{equation}
        then,
        \begin{equation}
            y_{1} - y_{2} = 0
        \end{equation}
    \end{theorem}
    \begin{proof}
        \begin{alignat}{1}
            L[y_{1}] & = g\\
            L[y_{2}] & = g
        \end{alignat}
        Then,
        \begin{equation}
            L[y_{1}] - L[y_{2}] = L[y_{1}, y_{2}] = g - g = 0
        \end{equation}
    \end{proof}
    \begin{theorem}[General Solutions to NSOLDEs]
        Given the following NSOLDE
        \begin{equation}
            \label{gsnsolde}
            L[y] = y'' + p(t)y' + q(t)y = g(t)
        \end{equation}
        and its corresponding HSOLDE,
        \begin{equation}
            \label{gshsolde}
            L[y] = y'' + p(t)y' + q(t)y = 0
        \end{equation}
        The general solution involves a particular solution of \textbf{Equation \ref{gsnsolde}} consists of a particular solution and a solution of the corresponding HSODLE,
        \begin{equation}
            \label{gs}
            y = \phi(t) = c_{1}y_{1} + c_{2}y_{2} + Y
        \end{equation}
        where $y_{1}, y_{2}$ are solutions to \textbf{Equation \ref{gshsolde}}, and $Y$ is a particular solution of \textbf{Equation \ref{gsnsolde}}
    \end{theorem}
    \begin{proof}
        From the \textbf{Theorem 13.1}, we get
        \begin{equation}
            \phi - Y = c_{1}y_{1} + c_{2}y_{2}
        \end{equation}
        this is the same as \textbf{Equation \ref{gs}}.
    \end{proof}
    From \textbf{Theorem  13.2}, solving NSOLDEs involves
    \begin{enumerate}
        \item Finding the general solution of the corresponding HSOLDE. (complementary solution; $y_c$)
        \item Find any particular solution, $Y$, to the NSOLDE.
        \item Then, the solution to the NSOLDE is the sum of the particular and general solutions.
    \end{enumerate}
    \np
    \section{Method of Undetermined Coefficients}
    \begin{example}
        Find a particular solution to
        \begin{equation}
            y''-3y'-4y=3e^{2t}
        \end{equation}
        The corresponding HSOLDE is given by
        \begin{equation}
            y'' - 3y' -4y = 0
        \end{equation}
        It is possible to "guess" the particular solution to a NSOLDE based on the form of $g(t)$. In this case, a reasonable guess would be
        \begin{equation}
            Y = Ae^{2t}
        \end{equation}
        Substituting $Y$ into the original equation yields
        \begin{equation}
            L[Ae^{2t}] = (Ae^{2t})'' - 3(Ae^{2t})' - 4(Ae^{2t}) = 3e^{2t}
        \end{equation}
        Solving for the undetermined coefficient,
        \begin{equation}
            (Ae^{2t})' = (2Ae^{2t}), (Ae^{2t})'' = (4Ae^{2t})
        \end{equation}
        \begin{alignat}{1}
            4Ae^{2t} - 3(2Ae^{2t}) - 4(Ae^{2t}) &= 3e^{2t}\\
            4A - 6A - 4A &= 3\\
            -6A &= 3\\
            A &= -\frac{1}{2}
        \end{alignat}
        Therefore,
        \begin{equation}
            Y = -\frac{1}{2}e^{2t}
        \end{equation}
        is a particular solution to this NSOLDE.
    \end{example}
    \np
    \begin{example}
        Find a particular solution of
        \begin{equation}
            y'' - 3y' - 4y = 2\sin(t)
        \end{equation}
        Guess,
        \begin{alignat}{1}
            Y &= A\sin(t)\\
            \implies  2\sin(t) &= -A\sin(t)-3A\cos(t)-6A\sin(t)
        \end{alignat}
        As can be seen above, the guess $A\sin(t)$ is incorrect due to the appearance of the $\cos(t)$ term, creating an open subspace. Creating a closed subspace yields
        \begin{equation}
            Y = A\cos(t) + B\sin(t)
        \end{equation}
        Then,
        \begin{alignat}{1}
            Y' &= -A\sin(t) + B\cos(t)\\
            Y'' &= -A\cos(t) + B\sin(t)\\
            L[Y] &= (-A\cos(t) + B\sin(t)) -3(-A\sin(t) + B\cos(t))\\
            &-4(A\cos(t) + B\sin(t)) = 2\sin(t)\\
            2\sin(t) &=\cos(t)(-A-3B-4A) + \sin(t)(-B+3A-4B)\\
            &\begin{cases}
                -A-3B-4A &= 0\\
                -B+3A-4B &= 2
            \end{cases}
        \end{alignat}
        By solving the linear combination,
        \begin{equation}
            A = \frac{3}{17},\quad B = -\frac{5}{17}
        \end{equation}
        Finally, a particular solution given by this method is
        \begin{equation}
            y = \frac{3}{17}\cos(t) = \frac{5}{7}\sin(t)
        \end{equation}
    \end{example}
    \np
    \begin{example}
        Find a particular solution of
        \begin{equation}
            y'' - 3y' -4y = -8e^{t}\cos(2t)
        \end{equation}
        A guess at a particular solution would be
        \begin{equation}
            Y = Ae^{t}\cos(2t) + Be^{t}\sin(2t)
        \end{equation}
        Then, substitution
        \begin{alignat}{1}
            Y' &= Ae^{t}\cos(2t)-2Ae^{t}\sin(2t) + Be^{t}\sin(2t) + 2Be^{t}\cos(2t)\\
            Y'' &= (-3A + 4B)e^{t}\cos(2t) - (4A + 3B)\sin(2t)\\
            & \begin{cases}
                10A + 2B &= 8\\
                2A-10B &= 0
            \end{cases}
        \end{alignat}
        Solving the linear combination yields
        \begin{equation}
            A = \frac{10}{13};\quad B = \frac{2}{13}
        \end{equation}
        Finally, a particular solution given by this method is
        \begin{equation}
            y = \frac{10}{13}e^{t}\cos(2t) + \frac{2}{13}e^{t}\sin(2t)
        \end{equation}
    \end{example}
    \np
    \begin{example}
        Find a particular solution of
        \begin{equation}
            \label{nsolde41}
            L[y] = y'' -3y' -4y = 3e^{2t} + 2\sin(t)
        \end{equation}
        It is possible to separate this NSOLDE into two,
        \begin{equation}
            \label{nsolde42}
            L[y] = 3e^{2t}; L[y] = 2\sin(t)
        \end{equation}
        And a particular solution to \textbf{Equation \ref{nsolde41}} is the linear combination of the solutions to the two NSOLDEs in \textbf{\ref{nsolde42}}.
        Having solved the two NSOLDEs previously,
        \begin{alignat}{1}
            Y_{1} &= -\frac{1}{2}e^{2t}\\
            Y_{2} &= -\frac{5}{17}\cos(t) + \frac{3}{17}\sin(t)
        \end{alignat}
        Then, a particular solution is given by
        \begin{equation}
            y = -\frac{1}{2}e^{2t} -\frac{5}{17}\cos(t) + \frac{3}{17}\sin(t)
        \end{equation}
    \end{example}
    \np
    \begin{example}
        Find a particular solution of
        \begin{equation}
            L[y] = y''- 3y' - 4y = 2e^{2t}
        \end{equation}
        if
        \begin{equation}
            Y = Ae^{-t};\quad Y' = -Ae^{-t};\quad Y'' = Ae^{-t}
        \end{equation}
        then,
        \begin{equation}
            L[Y] = Ae^{-t} - 3(-Ae^{-t}) - 4(Ae^{-t}) = 2e^{-t}
        \end{equation}
        When solving for A, we get $0 = 2$, which is false, therefore $Ae^{-t}$ is not a form of a particular solution.
        Then,
        \begin{equation}
            Y = Ate^{-t};\quad L[Y] = 2e^{-t}
        \end{equation}
        \begin{alignat}{1}
            L[Y] &= (Ate^{-t} - 2Ae^{-t})-3(Ae^{-t}-Ate^{-t}) - 4Ate^{-t}\\
            -5Ae^{t}&=2e^{-t}\\
            A &= -\frac{2}{5}\\
            y &= -\frac{2}{5}te^{-t}
        \end{alignat}
    \end{example}
    In the previous example, the corresponding homogeneous equation is
    \begin{equation}
        L[y] = y'' - 3y' - 4y = 0
    \end{equation}
    And the solutions to this equation are
    \begin{equation}
        y_{1} = e^{-t};\quad y_{2} = e^{4t}
    \end{equation}
    As it can be seen, the guess of $Y = Ae^{-t} = Ay_{1}$,
    \begin{equation}
        L[Ae^{-t}] = L[Ay_{1}] = AL[y_{1}] = 0
    \end{equation}
    Therefore,
    \begin{equation}
        L[Y] \neq 2e^{-t}
    \end{equation}
\end{document}