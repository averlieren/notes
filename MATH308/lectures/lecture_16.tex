% !TeX root = ../diffeq.tex
\documentclass[diffeq.tex]{subfiles}

\begin{document}
\chapter{24 March 2020}
    \section{Laplacians}
    \begin{equation}
        \lc \circ f(t) \longrightarrow F(s), f : \RR \to \RR
    \end{equation}
    Some uses for Laplacians include stochastic processes and probability.

    Note: As the inverse Laplace transform ($\lc^{-1}$) will not be introduced in this lecture, all steps involving $\lp{y} \to y$ will be skipped in the examples in this lecture.

    As defined previously,
    \begin{alignat*}{2} % TODO: Fix alignment
        \underset{\rotatebox{90}{$\,=$}}{\lp{f(t)}} &= \underset{\rotatebox{90}{$\,=$}}{\int_{0}^{\infty}e^{-st}f(t)\,dt}\\
        F(s) &= \lim_{x\to\infty}\int_{0}^{x}e^{-st}f(t)\,dt
    \end{alignat*}
    \begin{example}[Basic Examples]
        \begin{alignat}{1}
            f &\equiv 0\\
            \lp{f(t)} &= \int_{0}^{\infty}(e^{-st})(0)\,dt\\
            &= 0
        \end{alignat}
        \begin{alignat}{1}
            f &\equiv c\\
            \lp{f(t)} &= \int_{0}^{\infty}ce^{-st}\,dt\\
            &= \lim_{x\to\infty}\int_{0}^{x}ce^{-st}\,dt\\
            &= \lim_{x\to\infty}-\frac{c}{s}e^{-st}|_{0}^{x}\\
            &= \lim_{x\to\infty}\left(\frac{1}{s}e^{-st}\right)\\
            &= cs^{-1}
        \end{alignat}
    \end{example}
    \begin{btheorem}
        The Laplacian of a differential:
        \begin{alignat}{1}
            \lp{f'(t)} &= s\lp{f(t)} - f(0)\\
            \lp{f''(t)} &= s^{2}\lp{f(t)}-sf(0)-f'(0)
        \end{alignat}
    \end{btheorem}
    % TODO: Go back and add back missing notes.
    \begin{example}
        Find the solution of the following using Laplacians.
        \begin{equation}
            y'' - y' -2y = 0, y(0) = 1, y'(0) = 0
        \end{equation}
        Solution:
        \begin{alignat}{1}
            \lp{y'' - y' - 2y} &= \lp{0}\\
            \lp{y''} - \lp{y'} - 2\lp{y} &= 0\\
            (s^{2}\lp{y} - sy(0) - y'(0)) - (s\lp{y} - y(0)) - 2\lp{y} &= 0
        \end{alignat}
        \begin{alignat}{1}
            (s^{2} - s - 2)\lp{y} - s + 1 &= 0\\
            \lp{y} &= \frac{s - 1}{s^{2} - s - 2}\\
            &= \frac{\frac{1}{3}}{s-2} + \frac{\frac{2}{3}}{s+1}\\
            \implies y &= \frac{1}{3}e^{2t} + \frac{2}{3}e^{-t}
        \end{alignat}
    \end{example}
    \np
    \begin{example}
        Solve using Laplacians.
        \begin{equation}
            y'' + y = \sin(2t), y(0) = 2, y'(0) = 1
        \end{equation}
        \begin{alignat}{1}
            \lp{y'' + y} &= \lp{\sin(2t)}\\
            s^2\lp{y}-sy(0)-y'(0)+\lp{y} &= \frac{2}{s^{2} + 4}
        \end{alignat}
        \begin{alignat}{1}
            \lp{y} &= \frac{1}{s^{2} + 1}\left(\frac{1}{s^2 + 4} + 2s + 1\right)\\
            &= \frac{2s}{s^{2} + 1} + \frac{1}{s^{2} + 1} + \frac{2}{3}\left(\frac{1}{s^{2} + 1} - \frac{1}{s^{2} + 4}\right)
        \end{alignat}
        The full solution would involve the inverse Laplacian of the last equation (\textbf{16.25}).
    \end{example}
\end{document}