% !TeX root = ../diffeq.tex
\documentclass[diffeq.tex]{subfiles}

\begin{document}
\chapter{13 February 2020}
    \section{Applications of the Wronskian}
    \begin{corollary}
        Recall in the last lecture, \textbf{Theorem 9.3}.
        Let
        \begin{equation}
            A = \begin{pmatrix}
                a_{1} & b_{1}\\
                a_{2} & b_{2}
            \end{pmatrix}
        \end{equation}
        If $|A| = 0$, then for \textbf{some} values of $c_{1}$ and $c_{2}$ the linear system
        \begin{equation}
            \begin{cases}
                a_{1}x + b_{1}y = c_{1}&\\
                a_{2}x + b_{2}y = c_{2}&\\
            \end{cases}
        \end{equation}
        does not have a solution (inconsistent).
    \end{corollary}
    \np
    \begin{theorem}
        Assume that $y_{1}$ and $y_{2}$ are solutions to
        \begin{equation}
            L[y] = y'' + p(t)y' + q(t)y = 0
        \end{equation}
        where $p$ and $q$ are continuous, and $t_{0}$ is a fixed point.\\
        Then, $\forall z_{0} \wedge \forall z_{1}$, it is possible to find $c_{1}$ and $c_{2}$ such that
        \begin{equation}
            y(t) = c_{1}y_{1}(t) + c_{2}y_{2}(t)
        \end{equation}
        satisfies the IVP
        \begin{equation}
            L[y],\quad  y(t_{0}) = z_{0},\quad y'(t_{0}) = z_{1}
        \end{equation}
        if and only if
        \begin{equation}
            W[y_{1}, y_{2}](t_{0}) \neq 0
        \end{equation}
    \end{theorem}
    \begin{proof}
        Suppose that $\forall z_{0}, z_{1} \implies \exists\, c_{1}, c_{2}$ such that
        \begin{equation}
            \begin{cases}
                c_{1}y_{1}(t_{0}) + c_{2}y_{2(t_{0})} = z_{0}&\\
                c_{1}y_{1}'(t_{0}) + c_{2}y_{2}'(t_{0}) = z_{1}&
            \end{cases}
        \end{equation}
        then, $\exists!\,c_{1}, c_{2}$ iff $W[y_{1}, y_{2}](t_{0}) \neq 0$
    \end{proof}
    \begin{theorem}
        Suppose that $y_{1}$ and $y_{2}$ are solutions to
        \begin{equation}
            L[y] = 0
        \end{equation}
        Then, the family of solutions
        \begin{equation}
            y=c_{1}y_{1} + c_{2}y_{2}
        \end{equation}
        includes all solutions of $L[y] = 0$ iff $\exists t_{0} \implies W[y_{1}, y_{2}](t_{0}) \neq 0$
    \end{theorem}
    \np
    \begin{example}[Application of 10.2]
        The solutions to
        \begin{equation}
            y'' - 5y' + 6y = 0
        \end{equation}
        are
        \begin{equation}
            y_{1} = e^{2t},\quad y_{2} = e^{3t}
        \end{equation}
        \begin{equation}
            y = c_{1}e^{2t} + c_{2}e^{3t}
        \end{equation}
        Calculating the Wronskian
        \begin{alignat}{1}
            W[y_{1}, y_{2}] &= \begin{vmatrix}
                e^{2t} & e^{3t}\\
                2e^{2t} & 3e^{3t}
            \end{vmatrix}\\
            &= e^{5t}
        \end{alignat}
        \begin{equation}
            e^{5t} \neq 0
        \end{equation}
        Therefore, there does not exist other solutions to this CHSOLDE.
    \end{example}
    \begin{theorem}[Abel's Theorem]
        If $y_{1}, y_{2}$ are solutions to a SOLDE,
        \begin{equation}
            L[y] = y'' + p(t)y' + q(t)y = 0
        \end{equation}
        where $p, t$ are continuous over an open interval, $I$, then the Wronskian at point $t$ is given by Abel's Formula,
        \begin{equation}
            W[y_{1}, y_{2}](t) = c\exp\left(-\int p(t)\,dt\right)
        \end{equation}
        where $c$ is some arbitrary constant dependent on $y_{1}, y_{2}$, but not on $t$.
        \begin{alignat}{1}
            \forall t\in I, W[y_{1}, y_{2}](t) &\equiv 0 \iff c = 0\\
            \forall t\in I, W[y_{1}, y_{2}](t) &\nequiv 0 \iff c \neq 0
        \end{alignat}
    \end{theorem}
    \np
    \begin{example}
        Suppose that
        \begin{equation}
            y_{1}(t) = e^{r_{1}t},\quad y_{2}(t) = e^{r_{2}t}
        \end{equation}
        are solutions of
        \begin{equation}
            L[y] = y'' + p(t)y' + q(t)y = 0
        \end{equation}
        Show that if $r_{1} \neq r_{2}$, then $c_{1}y_{1} + c_{2}y_{2}$ includes all solutions of $L[y] = 0$.
        \begin{alignat}{1}
            W[e^{r_{1}t}, e^{r_{2}t}] &= \begin{vmatrix}
                e^{r_{1}t} & e^{r_{2}t}\\
                r_{1}e^{r_{1}t} & r_{2}e^{r_{2}t}
            \end{vmatrix}\\
            &= (r_{2} - r_{1})e^{(r_{1} + r_{2})t}\\
            &\neq 0
        \end{alignat}
    \end{example}
    \begin{definition}[Fundamental Set of Solutions]
        If $y_{1}$ and $y_{2}$ are solutions of
        \begin{equation}
            L[y] = y'' + p(t)y' + q(t)y = 0
        \end{equation}
        such that $c_{1}y_{1} + c_{2}y_{2}$ includes all possible solutions of $L[y] = 0$, then $y_{1}$ and $y_{2}$ form a \textbf{fundamental set of solutions} (FSS).\\
        Alternatively, if and only if
        \begin{equation}
            W[y_{1}, y_{2}] \neq 0
        \end{equation}
        then there exists fundamental set containing $y_{1}$ and $y_{2}$.
    \end{definition}
    \begin{example}
        Show that $y_{1}(t) = t^{\frac{1}{2}}$, $y_{2}(t) = t^{-1}$ form a FSS of
        \begin{equation}
            2t^{2}y'' + 3ty'-y=0,\quad t > 0
        \end{equation}
        \begin{enumerate}
            \item Ensure they are solutions of $L[y] = 0$\\
            i.e. $L[t^{\frac{1}{2}}] = 0$, $L[t^{-1}] = 0$
            \begin{alignat}{1}
                L[t^{\frac{1}{2}}] &= 2t^{2}(-\frac{1}{4})t^{-\frac{3}{4}}+3t(\frac{1}{2})t^{-\frac{1}{2}} + t^{\frac{1}{2}}\\
                &= -\frac{1}{2}t^{\frac{1}{2}} + \frac{3}{2}t^{\frac{1}{2}}-t^{\frac{1}{2}}\\
                &\equiv 0
            \end{alignat}
            \begin{alignat}{1}
                L[t^{-1}] &= (2t^{2})(2t^{-3}) + 3t(-1)t^{-2}-t^{-1}\\
                &= 4t^{-1} - 3t^{-1} - t^{-1}\\
                &\equiv 0
            \end{alignat}
            \item Ensure that the Wronskian is not constantly equal to 0
            \begin{alignat}{1}
                W[t^{\frac{1}{2}}, t^{-1}] &= \begin{vmatrix}
                    t^{\frac{1}{2}} & t^{-1}\\
                    \frac{1}{2}t^{-\frac{1}{2}} & -t^{-1}
                \end{vmatrix}\\
                &= -t^{-\frac{1}{2}} - -\frac{1}{2}t^{-\frac{3}{2}}\\
                &\nequiv 0
            \end{alignat}
        \end{enumerate}
        Then,
        \begin{equation}
            L[y] = y'' + ay' + by = 0;\quad f(r) = r^{2} + ar + b = 0
        \end{equation}
        has only one solution of degree 2.
        \begin{equation}
            r_{1,2} = \frac{-a\pm\sqrt{a^{2} - 4b}}{2},\quad r_{1} = r_{2} \iff \sqrt{a^{2} - 4b} \equiv 0
        \end{equation}
    \end{example}
    \begin{example}
        \begin{enumerate}
            \item If $ar^{2} + br + c = 0$ has equal roots $r_{1}$, show that
            \begin{equation}
                L[e^{rt}] = a(e^{rt})'' + b(e^{rt})' + ce^{rt} = a(r - r_{1})^{2}e^{rt}
            \end{equation}
            When $r = r_{1}$, $L[e^{rt}] = 0$, therefore $e^{rt}$ is a solution to
            \begin{equation}
                L[y] = ay'' + by' + cy = 0
            \end{equation}
            \item  Then,
            \begin{alignat}{1}
                \frac{\partial}{\partial r}L[e^{rt}] &= L[\frac{\partial}{\partial r}e^{rt}]=L[te^{rt}]\\
                &= ate^{rt}(r-r_{1})^{2} + 2ae^{rt}(r-r_{1})
            \end{alignat}
            Because $r = r_{1} \implies L[te^{rt}] = 0$, $te^{rt}$ is another solution to $L[y] = 0$.
        \end{enumerate}
        Show that $e^{rt}, te^{rt}$ form a FSS.
        \begin{enumerate}[label=\alph*.]
            \item $L[e^{rt}] = 0$
            \item $L[te^{rt}] = 0$
            \begin{alignat}{1}
                (te^{rt})' &= e^{rt} + rte^{rt}\\
                (te^{rt})'' &= 2re^{rt} + r^{2}te^{rt}\\
                %&= 2re^{rt} + r^{2}te^{rt} + b(te^{rt})\\
                L[te^{rt}] &= (2re^{rt} + r^{2}te^{rt}) + a(e^{rt} + rte^{rt}) + b\\
                &= e^{rt}(2r + a) + te^{rt}(r^{2} + ar + b)
            \end{alignat}
        \end{enumerate}
    \end{example}
\end{document}