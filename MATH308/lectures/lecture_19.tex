% !TeX root = ../diffeq.tex
\documentclass[diffeq.tex]{subfiles}

\begin{document}
\chapter{2 April 2020}
    \section{Laplacian Properties}
    Given functions $f, g$, then the Laplacian of $f + g$ can be written as the linear combination of the Laplacian of $f$ and $g$, i.e.:
    \begin{equation}
        \lp\{f + g\} = \lp\{f\} + \lp\{g\}
    \end{equation}
    However, does the following hold?
    \begin{equation}
        \lp\{f \cdot g\} = \lp\{f\} \cdot \lp\{g\}
    \end{equation}
    \begin{example}[Counter Example]
        Consider the case of
        \begin{equation}
            f = g = 1
        \end{equation}
        Then,
        \begin{equation}
            \lp\{f \cdot g\} = \lp\{1\} = \frac{1}{s}
        \end{equation}
        \begin{equation}
            \lp\{f\}\lp\{g\} = \frac{1}{s^{2}}
        \end{equation}
        \begin{equation}
            \frac{1}{s^{2}} \neq \frac{1}{s}
        \end{equation}
    \end{example}
    \np
    \section{Convolutions}
    However, there does exist a function $h$ such that
    \begin{equation}
        \lp\{h\} = \lp\{f\}\lp\{g\}
    \end{equation}
    \begin{theorem}
        \begin{equation}
            h(t) = \int_{0}^{t}f(\tau)g(t - \tau)d\tau
        \end{equation}
        Then,
        \begin{equation}
            \lp\{h\} = \lp\{f\}\lp\{g\}
        \end{equation}
        The function $h$ is defined to be the \textbf{convolution} of $f$ and $g$, denoted by $f * g$.
    \end{theorem}
    \subsection{Convolution Properties}
    \begin{alignat}{2}
        f * g &= g * f & \quad\text{commutivity}\\
        f * (g_{1} + g_{2}) &= f * g_{1} + f * g_{2} & \quad\text{distributivity}\\
        (f * g) * h &= f * (g * h) & \quad\text{associativity}\\
        0 * f &= f * 0 = 0&
    \end{alignat}
    \begin{example}
        Compute
        \begin{equation}
            \lpinv\{H(s)\},\quad H(s) = \frac{a}{s^{2}(s^{2} + a^{2})}
        \end{equation}
        Solution
        \begin{alignat}{1}
            H(s) &= \underbrace{\frac{1}{s^{2}}}_{F(s)} \cdot \underbrace{\frac{a}{s^{2} + a^{2}}}_{G(s)}\\
            \lpinv\{H(s)\} &= \lpinv\{F(s)\} * \lpinv\{G(s)\}\\
            &= \lpinv\left\{\frac{1}{s^{2}}\right\}*\lpinv\left\{\frac{a}{s^{2} + a^{2}}\right\}\\
            &= t * \sin(at)\\
            &= \int_{0}^{t}(t-\tau)\sin(a\tau)d\tau\\
            &= \frac{at-\sin(at)}{a^{2}}
        \end{alignat}
    \end{example}
    \begin{example}
        Solve
        \begin{equation}
            y'' + 4y = g(t),\quad y(0) = 3,\quad y'(0) = -1
        \end{equation}
        \begin{alignat}{1}
            \lp\{y'' + 4y\} &= \lp\{g\}\\
            s^{2}\lp\{y\} - sy(0) - y'(0) + 4\lp\{y\} &= \lp\{g\}
        \end{alignat}
        \begin{alignat}{1}
            \lp\{y\} &= 3\frac{s}{s^{2} + 4} - \frac{1}{2}\cdot\frac{2}{s^{2} + 4} + \frac{1}{2} \cdot \frac{2}{s^{2} + 4} \cdot G(s)\\
            y &= \lpinv\left\{3\frac{s}{s^{2} + 4} - \frac{1}{2}\cdot\frac{2}{s^{2} + 4}\right\} + \frac{1}{2}\lpinv\left\{\frac{2}{s^{2} + 4}G(s)\right\}\\
            &= 3\cos(2t)-\frac{1}{2}\sin(2t)+\frac{1}{2}(f * g)\\
            f(t) &= \lpinv\left\{\frac{2}{s^{2} + 4}\right\} = \sin(2t)
        \end{alignat}
        Then the final solution is,
        \begin{equation}
            y = 3\cos(2t) - \frac{1}{2}\sin(2t) + \frac{1}{2}\int_{0}^{t}\sin(2(t - \tau))g(\tau)\,d\tau
        \end{equation}
    \end{example}
    \begin{example}
        Find $\lp\{f\}$ given
        \begin{equation}
            f = \int_{0}^{t}\sin(t-\tau)\cos(\tau)\,d\tau
        \end{equation}
        Solution,
        \begin{alignat}{1}
            f &= \sin(t) * \cos(t)\\
            \lp\{f\} &= \lp\{\sin(t)\}\lp\{\cos(t)\}\\
            &= \frac{1}{s^{2} + 1}\cdot\frac{s}{s^{2} + 1}\\
            &= \frac{s}{(s^{2} + 1)^{2}}
        \end{alignat}
    \end{example}
\end{document}