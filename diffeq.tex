\documentclass[twoside]{report}
\usepackage{brandon}
\begin{document}
    \maketitle
    \chapter{First Order Differential Equations}
    \sss{14 January 2020}
    \begin{definition}
        The basic form of a \textbf{first order differential equation} is
        \[
            y' = f(x, y)
        \]
    \end{definition}
    \begin{definition}
        A solution is considered to be \textbf{general} if there is an arbitrary constant, $C$ in the final answer, i.e. a problem without an initial value.
    \end{definition}
    \begin{example}
        \begin{alignat}{1}
            y' &= 1\\
            y &= \int y'\,dx\\
            &= 1\,dx\\
            &= x + C
        \end{alignat}
    \end{example}
    \np
    \begin{definition}
        Equations without solutions are considered to be \textbf{open}. Many differential equations are without solutions.
    \end{definition}
    \begin{example}[Open Differential Equation]
        \begin{equation}
            y' = x'y - x^{3}
        \end{equation}
        This differential equation does not have a solution; thusly open.
    \end{example}
    \begin{example}
        \begin{equation}
            y' = y
        \end{equation}
        \begin{equation}
            \int y'\,dy = \int y\,dy \notLeftrightarrow y' = y
        \end{equation}
        Notice above that the integration of both sides is not the same as the differential equation.
        \begin{equation}
            y' = e^x \implies y \int y'\,dx = \int e^x\,dx
        \end{equation}
        Using the above, the general solution can be found
        \begin{equation}
            y = Ce^{x}
        \end{equation}
    \end{example}
    \begin{remark}[Regarding Example 1.3]
        If both sides of a differential equation are dependent on the same varible --- i.e. the same variable appearing on both sides of the equation, then taking the intergal of both sides is not a valid method to solve the equation.
    \end{remark}
    \begin{definition}
        An \textbf{initial value problem}, or \textbf{initial condition problem}, is a problem where an initial condition of the equation is defined which leads to a \textbf{unique solution} to the equation.
    \end{definition}
    \np
    \begin{example}[Initial Value Problem]
        \begin{equation}
            y' = x,\ y(0) = 1
        \end{equation}
        Notice that this is an \textbf{initial value problem}, because $y(0) = 1$. Also notice that $y$ is an anti-derivative w.r.t. $x$; because each side of the equation is independent of one another (unlike \emph{Example 1.3}).
        \begin{alignat}{3}
            &&\int y'\,dx &= \int x\,dx\\
            &\implies & y &= \frac{1}{2}x^{2} + C\\
            &&y(0) &= 1\\
            &\implies&=&\frac{1}{2}(0^2) + C\\
            &\implies&C&=1\\
            &\implies&y&=\frac{1}{2}x^{2} + 1
        \end{alignat}
    \end{example}
    \begin{definition}
        A function $f: \mathbb{R} \to \mathbb{R}$ is \textbf{differentiable} at point $a$ if $\exists\ T_1(a)$ where $T_1$ is the Taylor polynomial of the first degree. (Or, there exists a tangent line at $a$).
        \begin{center}
            \begin{tikzpicture}
                \draw[<->] (-1,0) -- (5,0) node[right] {$x$};
                \draw[<->] (0,-1) -- (0,5) node[above] {$y$};
                \draw[red,<->] (-1,1.625) --(1,1.625);
                \node[red,label={[xshift=5]A}] at (0,1.6) {\textbullet};
                \node[black,label={B}] at (2.6,.585) {\textbullet};
                \draw[scale = .65, domain=-1:4,smooth,variable=\x,blue] plot ({\x}, {-.1*\x*\x+2.5});
                \draw[scale = .65, domain=4:8,smooth,variable=\x,blue] plot ({\x}, {-.1*(\x-14)*(\x-14)+10.9});
            \end{tikzpicture}
        \end{center}
        In the example, point A has a singular tangent line and is therefore differentiable. Point B has infinitely many tangent lines, and is therefore both undefined and not differentiable.
    \end{definition}
    \np
    \begin{example}[Kinematics and Differential Equations]
        Given an object with a velocity $v_0$, and acceleration $a$, find the position $s$ at any time $t$.
        \begin{alignat}{3}
            &&\frac{d}{dt}v(t) &= a\\
            &\implies &v(t) &= \int a\,dt\\
            &&&= at + C\\
            &\because & v(0) &= v_{0}\\
            &&v_{0} &= a(0) + C\\
            &&C &= v_{0}\\
            &&\frac{d}{dt}s(t) &= \int v\,dt\\
            &\implies& s(t) &= \int v\,dt\\
            &&&= \int (at + v_{0})\,dt\\
            &&&= \frac{1}{2}a t^{2} + v_{0}t
        \end{alignat}
    \end{example}
    \chapter{Linear Differential Equations}
    \sss{16 January 2020}
    \begin{definition}
        \begin{equation}
            \underbrace{y' + p(t)y = g(t)}_\text{\tiny Usual form}\iff y' = g(t) - p(t)y
        \end{equation}
        A \textbf{linear differential equation} (LDE) is a first order differential equation due to $y$ being dependent on only one variable, $t$.
    \end{definition}
    \textbf{Note:} $t$ is typically used in place of $x$ as most differential equations are used in models dependent on time; as such, most differential equations are in the form $y' = f(t, y)$ as opposed to $y' = f(x, y)$.
    \begin{example}
        \begin{alignat}{3}
            &\text{Solve } &(4 + t^{2})y' + 2ty &= 4t\\
            &\text{Notice: }&(4y + t^{2}y)' &= \frac{d}{dt}(4y + t^{2}y) = 4t\\
            &&&= 4y' + (t^{2}y)^{2}\\
            &&&= 4y' + (2ty + t^{2}y')\\
            &&&= (4 + t^2)y' + 2ty
        \end{alignat}
        The original problem can now be reduced to:
        \begin{alignat}{3}
            &&\frac{d}{dt}(4y + t^2y)&= 4t\\
            &&\text{let\quad}z(t) &= 4y + t^{2}y\\
            &&&= 2 t^{2} + C\\
            &\implies& 4y + t^{2}y &= 2 t^{2}+C\\
            &\therefore& y &= \frac{1}{4+t^{2}}(2 t^{2}+C)
        \end{alignat}
    \end{example}
    \begin{remark}[Constants]
        Notice in the above example that the constant, $C$, is being multiplied by $\frac{1}{4 + t^{2}}$. When expanding the answer, it now becomes $y = \frac{2t^{2}}{4+t^{2}} + \frac{C}{4+t^{2}}$. Notice how the constant is dependent on the variable $t$, and is therefore not the same as just $C$.
    \end{remark}
    \begin{definition}
        An \textbf{integrating factor}, $\mu(t)$ is a function $\mu(t): \RR \to \RR$, that satisfies $\frac{d}{dt}\mu(t) = \mu(t)y'+\mu(t)p(t)y$.
    \end{definition}
    \begin{remark}
        There are infinitely many integrating factors due to the arbitrary constant $C$ from indefinite integration, see \textbf{Method 2.1} and \textbf{Example 2.2} on the following page.
    \end{remark}\np
    \begin{method}[Solution of the General Case]
        Solve $y' + p(t)y = g(t)$.
        \begin{enumerate}
            \item Multiply the LDE by $\mu(t)$ results in:
                \begin{equation}
                    \mu(t)(y' + p(t)y) = \mu(t)g(t)
                \end{equation}
            \item Letting $z(t) = \mu(t)y$, and $z' = \mu(t)g(t)$ yields:
                \begin{align}
                    z(t) &= \int \mu(t)g(t)\,dt\\
                    \implies y(t) &= \frac{1}{\mu(t)}\int \mu(t)g(t)\,dt\\
                    \implies \mu(t) &= \pow\BBp{\int p(t)\,dt}
                \end{align}
            \item Therefore the solution of the general case is
                \begin{equation}
                    y(t) = \BBp{\pow\Bp{\int p(t)\,dt}}^{-1}\cdot \int \pow\Bp{\int p(t)\,dt}g(t)\,dt
                \end{equation}
        \end{enumerate}
    \end{method}
    \begin{example}[Solving an Initial Value Problem]
        Working with example 2.1.4 from the textbook:
        \begin{equation}
            ty' +4 2y = 4 t^{2},\ y(1) = 2
        \end{equation}
        \begin{enumerate}
            \item Compute the integrating factor ($\mu(t)$)
                \begin{align}
                    \mu(t) &= \pow\bbbp{\int p(t)\,dt}\\
                    &= \pow\bbbp{\int 2 t^{-1}\,dt}\\
                    &= \pow\bp{2\ln(t) + C} \Leftrightarrow e^{2\ln(t) + C}
                \end{align}
            \item Find the general case\\
                When solving, $0$ can be subsituted in for $C$ to simplify calculations; for $C \neq 0$ it is trivially shown that the constant will cancel out in computing the solution.
                \begin{align}
                    y_c(t) &= \frac{1}{\mu(t)}\int \mu(t)g(t)\,dt\\
                    &= \frac{1}{t^{2}}\bbbp{\int t^{2}\cdot 4t\,dt}\\
                    &= \frac{1}{t^{2}}(t^{4} + C)
                \end{align}
                \textbf{Note:} $y_c(t)$ is used to denote the general case.
            \item Find formula w.r.t. intial value
                \begin{alignat}{3}
                    &&y(1) &= 2\\
                    &\implies &y(1) &= (1)^{2} + \frac{C}{(1)^{2}}\\
                    &\implies &C &= 1\\
                    &\therefore &y(t) &= t^{2} + t^{-2}
                \end{alignat}
        \end{enumerate}
    \end{example}
    \chapter{Separable Differential Equations}
    \sss{21 January 2020}
    More LDE examples
    \begin{example}
        Given $y'-2y=t^{2}e^{2t}$ find:
        \begin{enumerate}
            \item The general solution
            \begin{align}
                p(t) = -2,&\ g(t) = t^{2}e^{2t}\\
                \mu(t) %&= \pow\bbbp{\int p(t)\,dt}\\
                &= \pow\bbbp{\int -2\,dt}\\
                &= e^{-2t+C}\\
                y_c(t) &= e^{2t}\int t^{2}\,dt\\
                &= e^{2t}\BBp{\frac{1}{3}t^{3} + C}
            \end{align}
            \item What is $\lim_{t \rightarrow \infty} y_c(t)$?\\
            There are infinitely many $y_c(t)$; the answer may vary with the value of $C$. In this case, the value of $C$ does not matter.
            $$\lim_{t \rightarrow \infty} y_{c}(t) = +\infty$$
        \end{enumerate}
    \end{example}
    \np
    \begin{definition}
        A \textbf{separable differential equation} can be defined by
        \begin{equation}
            \frac{dy}{dx}=y'=f(x, y)=-\frac{M(x, y)}{N(x, y)}
        \end{equation}
        where
        \begin{align}
            M(x, y) &= - f(x, y)\\
            N(x, y) &= 1
        \end{align}
        it is \textbf{separable} because it can be written in the \textbf{differential form}
        \begin{equation}
            M(x)\,dx+N(y)\,dy=0
        \end{equation}
    \end{definition}
    \begin{btheorem}
        If $\frac{dy}{dx} = \frac{M(x)}{N(y)}$, then $\int N(y)\,dy=\int M(x)\,dx$\\
        \textbf{Proof:} Choose $\widetilde{N}$ such that $\frac{d\widetilde{N}(y)}{dx} = M(x)$:
        \begin{align}
            \frac{d\widetilde{N}(y)}{dy} = \frac{d\widetilde{N}(y)}{dx}\frac{dx}{dy} &= \frac{d\widetilde{N}(y)}{dy}\frac{dy}{dx}= \frac{d\widetilde{N}(x)}{dx}\\
            \frac{d\widetilde{N}(y)}{dy} &= \frac{dy}{dx}
            \\
            \implies \frac{d\widetilde{N}(y)}{dx} &= M(x)
        \end{align}
    \end{btheorem}
    \np
    \begin{example}
        Find a particular solution that passes through the point $(0, 1)$.
        \begin{align}
            \frac{dy}{dx} &= \frac{4x-x^{3}}{4 + y}\\
            \implies \int (4 + y)\,dy &= \int(4x-x^{3})\,dx\\
            4y+\frac{1}{2}y^{2} + C_1 &= 2 x^{2}-\frac{1}{4}x^{4} + C_2\\
            4y + \frac{1}{2} y^{2} &= 2 x^{2} - \frac{1}{4} x^{4} + (C_2 - C_1)\\
            \implies 2y + 16y + x^{4} &- 8x^{2} + C = 0\\
            (0, 1) \implies 2(1) + 16(1) + 0^{4} &- 8(0)^{2} + C = 0\\
            C &= -18\\
            \therefore 2y + 16y &+ x^{4} - 8x^{2} = 18
        \end{align}
    \end{example}
    \begin{homework}
        \begin{align}
            y' = \frac{dy}{dx} &= \frac{x^{2}}{y}\\
            y\,dy &= x^{2}\,dx\\
            \int y\,dy &= \int x^{2}\,dx\\
            \frac{1}{2}y^{2} &= \frac{1}{3}x^{3} + C\\
            y(x) &= \pm \sqrt{\frac{2}{3}x^{3} + C}
        \end{align}
    \end{homework}
    \chapter{Applications of Mathematical Modelling}
    \sss{23 January 2020}
    More separable equation examples
    \begin{example}
        From the textbook, 2.2, ex. 2.
        \begin{alignat}{2}
            \frac{dy}{dx}&=\frac{3x^2+4x+2}{2(y-1)} &\quad y(0) = -1
        \end{alignat}
        Given the above, determine the interval in which the solution exists.
        \begin{align}
            \int 2(y-1)\,dy&=\int (3x^{2} + 4x + 2)\,dx\\
            \implies y^2 - 2y + C_{1} &= x^3 + 2x^{2}+2x+C_2
        \end{align}
        The solution above is the \textbf{general implicit solution}. The constants, $C_{1}$ and $C_2$ can be combined into one constant, $C$, because they are independent.\\
        Next, use the initial value to solve for C
        \begin{alignat}{3}
            &&y(0) &= - 1\\
            &\implies & (-1)^{2} - 2(-1) &= 0^{3} + 2(0)^{2}+ 2(0) + C\\
            &\implies & C &= 3
        \end{alignat}
    \end{example}
    \np
    \begin{example*}[4.1 (cont.)]
        Then complete the square on the left hand side to get the \textbf{explicit~solution}.
        \begin{alignat}{3}
            && (y^{2} - 2y + 1) - 1 &= x^{3} + 2x^{2} + 2x + 3\\
            &\implies &(y - 1)^{2} &= x^{3} + 2x^{2} + 2x + 4\\
            &\implies & y - 1 &= \pm \sqrt{x^{3}  + 2x^{2} + 2x + 4}\\
            &\implies & y &= 1 \pm \sqrt{x^{3}  + 2x^{2} + 2x + 4}\\
            &\implies & y &= 1 - \sqrt{x^{3}  + 2x^{2} + 2x + 4}\\
            &\because & y(0) &= -1
        \end{alignat}
        \textbf{Note:} It is also possible to use the quadratic formula in order to convert this instance of an implicit into an explicit solution.\\
        \textbf{Observation}: Because the unique solution involves a square root, a function defined for $x \in [0, \infty)$, it is possible to reduce the original question to findinding when the radicand is non-negative.
        \begin{alignat}{3}
            &&x^{3} + 2 x^{2} + 2x + 4\ &= 0\\
            &&(x^{2} + 2)(x + 2) &= 0\\
            &\implies&x &\geq -2
        \end{alignat}
        The factor $x^{2} + 2$ will always be positive, so now the question is further reduced to when $x + 2$ will be non-negative, which is $x \in [-2, \infty)$.\\
        Therefore, the interval of which the solution exists is $(-2, \infty)$
    \end{example*}
    \begin{remark}[Solutions to Differential Equations]
        In \textbf{Example 4.1}, notice the final answer was an open interval, $(-2, \infty)$, rather than a half closed interval, $[-2, \infty)$, even if the solution would be defined if $x = -2$. The reason for this is that \textbf{solutions to differential equations must also be differentiable}.\\
        At point $x = -2$, the unique solution is defined, however, it is not differentiable as $\lim_{x \to -2^{-}}$ does not exist, because the function is not defined for $x < -2$.
    \end{remark}
    \np
    \begin{example}[Modelling]
        Consider a pond fille with 10 million gallons of fresh water. A flow of 5 million gallons per year with water that is contaminated wiht a chemical enters the pond. There is also an outflow of this mixture on the order of 5 million gallons per year.\bigskip

        \noindent Let $\gamma(t)$ be the concentration of the fluid entering the pod at time $t$, and let $Q(t)$ be the quantity of chemicals in the pod at time $t$.\bigskip
        
        \noindent It is determined that
        \[\gamma(t) = 2 + \sin(2t)\ \text{g}\cdot\text{gal}^{-1}\]

        \noindent Find $Q(t)$ using the given information.\bigskip

        \noindent We can infer that $Q(0) = 0$ because the water starts off fresh at $t = 0$.\\
        We know that $\frac{dQ}{dt}$ is equal to the rate at which chemicals are entering minus the rate at which they leave, leading us to
        \[\frac{dQ}{dt} = I(t)\gamma(t) - \frac{O(t)}{V(t)}\big[Q(t)\big]\]
        Where $I(t)$ describes the rate at which the contaminated water enters, $O(t)$ describes the rate at which the water mixture leaves the pond, and $V(t)$ describes the total volume of the pond at any given time.\\
        In this case,
        \begin{alignat}{1}
            I(t) &= 5\times 10^{6}\ \text{gal}\,\text{year}^{-1}\\
            O(t) &= 5\times 10^{6}\ \text{gal}\,\text{year}^{-1}\\
            V(t) &= 10^{7}\ \text{gal}\\
        \end{alignat}
        Plugging in the values yields the following,
        \begin{alignat}{1}
            \frac{dQ}{dt} &= 5\times 10^{6}\gamma(t) - \frac{1}{2}Q(t)\\
        \end{alignat}
        Solving the linear differential equation,
        \begin{alignat}{3}
            &&\frac{dQ}{dt} + \frac{1}{2}Q(t) &=  5\times 10^{6}\gamma(t)\\
            &\implies&Q_{c}(t)&=5\times 10^{6}e^{-\frac{1}{2}t}\int e^{\frac{1}{2}t}(2 + \sin(2t))\,dt\\
            &\implies&Q_{c}(t)&=2\times 10^{7} + \frac{2\times 10^{7}}{17}\sin(2t)-\frac{4\times 10^{7}}{17}cos(2t) + Ce^{-\frac{1}{2}t}\\
            &&Q_{c}(0) &= 2\times 10^{7}-\frac{4\times 10^{7}}{17} + C = 0\\
            &\implies&C&=\frac{-3\cdot10^{8}}{17}
        \end{alignat}
        \begin{equation}    
            Q(t)=2\times 10^{7} + \frac{2\times 10^{7}}{17}\sin(2t)-\frac{4\times 10^{7}}{17}cos(2t) - \frac{3\cdot10^{8}}{17}e^{-\frac{1}{2}t}
        \end{equation}
    \end{example}
    \begin{remark}[Behavior of Example 4.2]
        When graphing this equation, it can be seen that in the long term the equation becomes periodic despite beginning with an irregular pattern. This is due to the fact that the term $- \frac{3\cdot10^{8}}{17}e^{-\frac{1}{2}t}$ is able to affect the behavior in the short term, however, it is decaying exponentially and tends towards $0$. The $\sin$ and $\cos$ functions are periodic which cause the sinusoidial shape of the graph as $t \to \infty$.
    \end{remark}
    \chapter{TBD}
    \sss{28 January 2020}
    \chapter{TBD}
    \sss{30 January 2020}
\end{document}